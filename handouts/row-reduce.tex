\documentclass{homework}
\usepackage{amsmath}
\DeclareMathOperator{\Mat}{Mat}
\DeclareMathOperator{\End}{End}
\DeclareMathOperator{\Hom}{Hom}
\DeclareMathOperator{\id}{id}
\DeclareMathOperator{\image}{im}
\DeclareMathOperator{\Imag}{Imag}
\DeclareMathOperator{\rank}{rank}
\DeclareMathOperator{\nullity}{nullity}
\DeclareMathOperator{\trace}{tr}
\DeclareMathOperator{\Spec}{Spec}
\DeclareMathOperator{\Sym}{Sym}
\DeclareMathOperator{\pf}{pf}
\DeclareMathOperator{\Ortho}{O}

\newcommand{\C}{\mathbb{C}}

\DeclareMathOperator{\sla}{\mathfrak{sl}}
\newcommand{\norm}[1]{\left\lVert#1\right\rVert}
\newcommand{\transpose}{\intercal}

\usepackage{hyperref}
\author{Jim Fowler}
\course{Math 5520H}
\title{Can you row-reduce?}
\date{Thursday, August 22, 2019}
\begin{document}
\maketitle

This course---and mathematics generally!---is primarily about
concepts, but I also have some computational goals for you.

\section*{Some row reduction}

\begin{problem}
  Find a matrix row-equivalent to
\(\begin{bmatrix}
  1 & 2 & 3 \\
  4 & 5 & 6 \\
  7 & 8 & 9
\end{bmatrix}\) and in reduced row echelon form.
\end{problem}

\vfill

\begin{problem}
  What is the rank of
  \(\begin{bmatrix}
    1 & 2 & 1 & 6 \\
    3 & 6 & 1 & 14 \\
    1 & 2 & 2 & 8
  \end{bmatrix}\)?
\end{problem}

\vfill

\section*{More complicated row reduction}

\begin{problem}
  For real numbers $x$ and $y$, put
\(\begin{bmatrix}
  1 & x \\
  1 & y 
\end{bmatrix}\) in reduced row echelon form.
\end{problem}

\vfill

\begin{problem}
  Suppose $a \neq 0$ and $b$ and $c$ are real numbers.  When is
\(\begin{bmatrix}
  a & b \\
  c & 1
\end{bmatrix}\) row equivalent to the identity matrix?
\end{problem}

\vfill

\section*{Homogeneous systems}

\begin{problem}
  A certain homogeneous system of linear equations consists of 50
  equations in 100 variables.  How many solutions might it have?
\end{problem}

\begin{problem}
  A homogeneous system of linear equations consists of 100 equations
  in 50 variables.  How many solutions might it have?
\end{problem}

\section*{Finding solution sets}

\begin{problem}
  Write down a system of linear equations with \textit{no} solutions.
\end{problem}

\begin{problem}
  What are the possible sizes of the solution set for a system of linear
  equations?
\end{problem}

\begin{problem}
  You roll a D20 nine times to build a homogeneous system of linear
  equations with three equations and three unknowns.  How many
  solutions do you expect it to have?
\end{problem}


\end{document}