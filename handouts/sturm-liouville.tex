\documentclass{homework}
\usepackage{amsmath}
\DeclareMathOperator{\Mat}{Mat}
\DeclareMathOperator{\End}{End}
\DeclareMathOperator{\Hom}{Hom}
\DeclareMathOperator{\id}{id}
\DeclareMathOperator{\image}{im}
\DeclareMathOperator{\Imag}{Imag}
\DeclareMathOperator{\rank}{rank}
\DeclareMathOperator{\nullity}{nullity}
\DeclareMathOperator{\trace}{tr}
\DeclareMathOperator{\Spec}{Spec}
\DeclareMathOperator{\Sym}{Sym}
\DeclareMathOperator{\pf}{pf}
\DeclareMathOperator{\Ortho}{O}

\newcommand{\C}{\mathbb{C}}

\DeclareMathOperator{\sla}{\mathfrak{sl}}
\newcommand{\norm}[1]{\left\lVert#1\right\rVert}
\newcommand{\transpose}{\intercal}

\usepackage{hyperref}
\author{Jim Fowler}
\course{Math 5520H}
\title{The tiniest bit of Sturm-Liouville}
\date{October 31, 2019}
\newcommand{\C}{\mathbb{C}}
\DeclareMathOperator{\image}{im}
\begin{document}
\maketitle

Learning is about making connections.  As usual, we relate the
differential equation story and the linear algebra story.

\begin{problem}
  Consider the operator $D = -\frac{d^2}{dx^2}$ with domain consisting
  of smooth functions $f$ on $[0.2\pi]$ with the property that
  $f(0) = f(2\pi) = 0$.  What is the adjoint of $D$?
\end{problem}

\vfill

\begin{problem}
  The eigenvalues of a self-adjoint operator are real.
\end{problem}

\vfill

\begin{problem}
  Eigenvectors of a self-adjoint operator, corresponding to distinct eigenvalues, are orthogonal.
\end{problem}

\vfill

\begin{problem}
  Sturm-Liouville theory studies differential equations of the form
\[
  \frac{d}{dx} \left[p(x){ \frac{d}{dx} y(x) }\right]+q(x)y=-\lambda w(x)y.
\]

  The \textbf{Legendre differential equation} is
  \[
    (1-x^2) y''(x) - 2x y'(x) + n(n+1) y(x) = 0.
  \]
  Put this into Sturm-Liouville form.
\end{problem}

\vfill

\begin{problem}
  Let $Ly = \left( (1-x^2) y' \right)'$, and compute
  \[
    \langle Lf, g \rangle - \langle f, Lg \rangle.
  \]
\end{problem}

\vfill

\end{document}
