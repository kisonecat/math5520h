\documentclass{homework}
\usepackage{amsmath}
\DeclareMathOperator{\Mat}{Mat}
\DeclareMathOperator{\End}{End}
\DeclareMathOperator{\Hom}{Hom}
\DeclareMathOperator{\id}{id}
\DeclareMathOperator{\image}{im}
\DeclareMathOperator{\Imag}{Imag}
\DeclareMathOperator{\rank}{rank}
\DeclareMathOperator{\nullity}{nullity}
\DeclareMathOperator{\trace}{tr}
\DeclareMathOperator{\Spec}{Spec}
\DeclareMathOperator{\Sym}{Sym}
\DeclareMathOperator{\pf}{pf}
\DeclareMathOperator{\Ortho}{O}

\newcommand{\C}{\mathbb{C}}

\DeclareMathOperator{\sla}{\mathfrak{sl}}
\newcommand{\norm}[1]{\left\lVert#1\right\rVert}
\newcommand{\transpose}{\intercal}

\usepackage{hyperref}
\author{Jim Fowler}
\course{Math 5520H}
\title{Tensor product}
\date{November 26, 2019}
\DeclareMathOperator{\Span}{span}
\DeclareMathOperator{\Hom}{Hom}
\begin{document}
\maketitle

Throughout, $V = \Span \{ v_1, \ldots, v_n \}$ and $W = \Span \{ w_1, \ldots, w_m \}$.  

\section{Prove or disprove and salvage if possible.}

\begin{problem}
  $0 \otimes v = 0$.
\end{problem}

\vfill

\begin{problem}
  If $v \in V$ and $w \in W$, then $v \otimes w$ is in $\Span \{ v_i \otimes w_j \}$.
\end{problem}

\vfill

\begin{problem}
  There exists a bilinear map $f_{k,\ell} :V \times W \to k$ so that \[
    f_{k,\ell}( \sum_i \alpha_i v_i, \sum_j \beta_j w_j) = \alpha_k \beta_\ell.
  \]
\end{problem}

\vfill

\begin{problem}
  The spanning set $\{ v_i \otimes w_j \}$ is linearly independent.
\end{problem}

\vfill

\begin{problem}
  Every element of $V \otimes W$ can be written as $v \otimes w$ for some $v \in V$ and $w \in W$.
\end{problem}

\vfill

\begin{problem}
  $V \otimes W \cong W \otimes V$.
\end{problem}

\vfill

\begin{problem}
  $(V \otimes W) \otimes U \cong V \otimes (W \otimes U)$.
\end{problem}

\vfill

\begin{problem}
  $V \otimes (W \oplus U) \cong (V \otimes W) \oplus U$.
\end{problem}

\vfill

\begin{problem}
  $V \otimes W \cong \Hom(V,W)$.
\end{problem}

\vfill

\section{Exploration.}

\begin{problem}
  Use P09 to define $V \otimes W$ for finite dimensional vector spaces.
\end{problem}

\vfill

\begin{problem}
  Describe an evaluation map $V \otimes V^\star \to k$.
\end{problem}

\vfill

\begin{problem}
  Suppose $\{ v'_1,\ldots, v'_n \}$ is another basis for $V$, and suppose there are basis $\{ f_1,\ldots,f_n\}$ and $\{ f'_1,\ldots,f'_n \}$ for $V^\star$, dual to $\{v_1,\ldots,v_n\}$ and $\{v'_1,\ldots,v'_n\}$ respectively.  Relate the ``change of basis'' matrix for the bases $\{v_1,\ldots,v_n\}$ and $\{v'_1,\ldots,v'_n\}$ to that of the duals.
\end{problem}

\vfill

\begin{problem}
  Describe a coevaluation map $k \to V \otimes V^\star$ by defining an element of $V \otimes V^\star$ via the formula $\sum v_i \otimes f_i$ where $\{ f_i \}$ is  a basis dual to $\{ v_i \}$.  How does this coevaluation map depend on that choice of basis?
\end{problem}

\vfill

\begin{problem}
  Regard $\mathbb{C}$ as a two-dimensional vector space over $\mathbb{R}$.  If $V$ is an $\mathbb{R}$-vector space, consider the complexification $V^{\mathbb{C}} := \mathbb{C} \otimes_{\mathbb{R}} V$, where ``$\otimes_{\mathbb{R}}$'' means we take the tensor product of $\mathbb{C}$ and $V$ as real vector spaces.  Can you describe a $\mathbb{C}$-vector space structure on $V^{\mathbb{C}}$?
\end{problem}

\vfill

\end{document}

%%% Local Variables:
%%% mode: latex
%%% TeX-master: t
%%% End:
