\documentclass{homework}
\usepackage{amsmath}
\DeclareMathOperator{\Mat}{Mat}
\DeclareMathOperator{\End}{End}
\DeclareMathOperator{\Hom}{Hom}
\DeclareMathOperator{\id}{id}
\DeclareMathOperator{\image}{im}
\DeclareMathOperator{\Imag}{Imag}
\DeclareMathOperator{\rank}{rank}
\DeclareMathOperator{\nullity}{nullity}
\DeclareMathOperator{\trace}{tr}
\DeclareMathOperator{\Spec}{Spec}
\DeclareMathOperator{\Sym}{Sym}
\DeclareMathOperator{\pf}{pf}
\DeclareMathOperator{\Ortho}{O}

\newcommand{\C}{\mathbb{C}}

\DeclareMathOperator{\sla}{\mathfrak{sl}}
\newcommand{\norm}[1]{\left\lVert#1\right\rVert}
\newcommand{\transpose}{\intercal}

\usepackage{hyperref}
\author{Jim Fowler}
\course{Math 5520H}
\title{Exactness}
\date{Friday, August 30, 2019}
\begin{document}
\maketitle

Suppose $f(x,y) = -P(x,y)/Q(x,y)$.  Rewrite $y' = f(x,y)$ as
$\omega = P \, dx + Q \, dy = 0$.  This form is \textbf{exact} if
there exists $F$ so that $dF = \omega$.

\begin{problem}
  Compute $\frac{d}{dx} F(x, y(x))$.
\end{problem}

\vfill

\begin{problem}
  Compute $d\omega$.  When $d\omega = 0$, we say $\omega$ is \textbf{closed}.
\end{problem}

\vfill

\begin{problem}
  Is $d$ linear?  What is the image of $d$?  The kernel of $d$?
\end{problem}

\vfill

\begin{problem}
  Is $y \, dx - x \, dy$ an exact differential?
\end{problem}

\vfill

\begin{problem}
  Relate $F$ to the solutions to $y' = f(x,y)$.
\end{problem}

\vfill

\begin{problem}
  If $d\omega = 0$, does there exist $F$ so that $dF = \omega$?
\end{problem}

\vfill

\begin{problem}
  Why is $F$ called a \textbf{potential}?
\end{problem}

\vfill

\begin{problem}
  If $\omega$ is not exact, it may nevertheless be that there is a $\mu$ so that $\mu P \, dx + \mu Q \, dy$ is exact.  In this case, $\mu$ is called an \textbf{integrating factor}.

  For $\omega = A(x) \, y \, dx + B(x) \, dx + dy$.  Set $\mu  = e^{\int A(x) \, dx}$.  Is $\mu \omega$ exact?
\end{problem}
% d(\mu \omega) = (d\mu)\omega + \mu d\omega =  (A \mu dx)(dy) + \mu (A dy_

\vfill

\begin{problem}
  Find $y$ so that $(3/x) y \, dx + (e^x / x^3) \, dx + dy = 0$.
\end{problem}
% integrating factor is exp(3 \log x) = x^3, so
% $3 x^2 y \, dx + e^x \, dx + x^3 dy = 0$.
% so d(e^x + x^3 y) = 0
% so e^x + x^3 y = c
% so y = (c - e^x)/x^3

\vfill

\begin{problem}
  Let us relate the preceding back to the \textbf{autonomous} case.  Wanting to solve $y' = f(x,y) = -P(x,y)/Q(x,y)$ amounts to finding $x(t)$ and $y(t)$ satisfying $x'(t) = Q(x,y)$ and $y'(t) = -P(x,y)$.  What role does the integrating factor play?
\end{problem}

\end{document}
