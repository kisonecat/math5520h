\documentclass{homework}
\usepackage{amsmath}
\DeclareMathOperator{\Mat}{Mat}
\DeclareMathOperator{\End}{End}
\DeclareMathOperator{\Hom}{Hom}
\DeclareMathOperator{\id}{id}
\DeclareMathOperator{\image}{im}
\DeclareMathOperator{\Imag}{Imag}
\DeclareMathOperator{\rank}{rank}
\DeclareMathOperator{\nullity}{nullity}
\DeclareMathOperator{\trace}{tr}
\DeclareMathOperator{\Spec}{Spec}
\DeclareMathOperator{\Sym}{Sym}
\DeclareMathOperator{\pf}{pf}
\DeclareMathOperator{\Ortho}{O}

\newcommand{\C}{\mathbb{C}}

\DeclareMathOperator{\sla}{\mathfrak{sl}}
\newcommand{\norm}[1]{\left\lVert#1\right\rVert}
\newcommand{\transpose}{\intercal}

\usepackage{hyperref}
\author{Jim Fowler}
\course{Math 5520H}
\title{Generating functions}
\date{November 1, 2019}
\newcommand{\C}{\mathbb{C}}
\DeclareMathOperator{\image}{im}
\begin{document}
\maketitle

\begin{problem}
  Write \(e^{(x/2)(t - 1/t)}\) as a series
  \[
    \sum_{n=-\infty}^\infty \left( \sum_{k=0}^\infty \frac{(-1)^k \, (x/2)^{2k + n}}{k! \, \Gamma( n + k + 1)}  \right) t^n.
  \]
\end{problem}

\vfill

\begin{problem}
  $J_n(-x) = J_{-n}(x)$.
\end{problem}

\vfill

\begin{problem}
  $J'_n(x) = \frac{J_{n-1}(x) - J_{n+1}(x)}{2}$.
\end{problem}

\vfill

\begin{problem}
  $2 n \, J_n(x) = x \, J_{n-1}(x) + x \,  J_{n+1}(x)$.
\end{problem}

\vfill

\begin{problem}
  $e^{i x \sin \theta} = \sum_{n=-\infty}^\infty J_n(x) e^{i n \theta}$.  This is the \textbf{Jacobi-Anger expansion.}
\end{problem}

\vfill

\end{document}

\begin{problem}
Recall that we found a series for $J_n(x)$, the Bessel function of the first kind of order $n$, namely
\[
  J_n(x) = .
\]
\end{problem}

\begin{problem}
  Suppose $x \neq 0$ is a zero of $J_n$.  Then $x$ is a simple zero.
\end{problem}

\begin{problem}
  If $J_n(x) = 0$, then $J_n(-x) = 0$.
\end{problem}

\begin{problem}
  If $J_n(x) = 0$, then $J_n(\bar{x}) = 0$.
\end{problem}

\begin{problem}
  Differentiate
  \[
    x \left( b J_n(ax) J'_n(bx) - a J'_n(ax) J_n(bx) \right)
  \]
  to compute
   $(a^2-b^2) \int_0^x t J_n(at) J_n(bt) \, dt$.
\end{problem}

\vfill

\begin{problem}
  Apply the previous computation to the case that  $a$ and $b$ are conjugate zeros of $J_n(x)$.
\end{problem}

\vfill

\begin{problem}
  What can you conclude about the zeroes of $J_n$?
\end{problem}

\vfill

\end{document}
