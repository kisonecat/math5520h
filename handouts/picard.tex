\documentclass{homework}
\usepackage{amsmath}
\DeclareMathOperator{\Mat}{Mat}
\DeclareMathOperator{\End}{End}
\DeclareMathOperator{\Hom}{Hom}
\DeclareMathOperator{\id}{id}
\DeclareMathOperator{\image}{im}
\DeclareMathOperator{\Imag}{Imag}
\DeclareMathOperator{\rank}{rank}
\DeclareMathOperator{\nullity}{nullity}
\DeclareMathOperator{\trace}{tr}
\DeclareMathOperator{\Spec}{Spec}
\DeclareMathOperator{\Sym}{Sym}
\DeclareMathOperator{\pf}{pf}
\DeclareMathOperator{\Ortho}{O}

\newcommand{\C}{\mathbb{C}}

\DeclareMathOperator{\sla}{\mathfrak{sl}}
\newcommand{\norm}[1]{\left\lVert#1\right\rVert}
\newcommand{\transpose}{\intercal}

\usepackage{hyperref}
\author{Jim Fowler}
\course{Math 5520H}
\title{Picard–-Lindel\"of theorem}
\date{October 2, 2019}
\begin{document}
\maketitle

\textbf{Picard--Lindel\"of Theorem.}

\textit{Proof.}

\begin{problem}
  If $f^n(x) = x$ and $f^n(y) = y$ implies $x = y$, then $f(x) = x$ and $f(y) = y$ implies $x = y$.
\end{problem}
% if f(x) = x, then f^n(x) = x

\begin{problem}
  If $f^n$ has a unique fixed point, then $f$ has a fixed point.
\end{problem}
% suppose $f^n(x) = x$.  Then $f^n(f(x)) = f(f^n(x)) = f(x)$.  So $f(x)$ is a fixed point of $f^n$, so $f(x) = x$.

Throughout, we will solve $y' = f(x,y)$ with $y(0) = 0$.
Suppose $U$ is an open set containing the origin, and $f : U \to \mathbb{R}$ is continuous.  Morevoer there exists $K > 0$ so that $|f(x,y_1) - f(x,y_2)| \leq K \cdot | y_1 - y_2 |$.

\begin{problem}
    For any $(t_0,y_0)
\end{problem}

\end{document}