\documentclass{homework}
\usepackage{amsmath}
\DeclareMathOperator{\Mat}{Mat}
\DeclareMathOperator{\End}{End}
\DeclareMathOperator{\Hom}{Hom}
\DeclareMathOperator{\id}{id}
\DeclareMathOperator{\image}{im}
\DeclareMathOperator{\Imag}{Imag}
\DeclareMathOperator{\rank}{rank}
\DeclareMathOperator{\nullity}{nullity}
\DeclareMathOperator{\trace}{tr}
\DeclareMathOperator{\Spec}{Spec}
\DeclareMathOperator{\Sym}{Sym}
\DeclareMathOperator{\pf}{pf}
\DeclareMathOperator{\Ortho}{O}

\newcommand{\C}{\mathbb{C}}

\DeclareMathOperator{\sla}{\mathfrak{sl}}
\newcommand{\norm}[1]{\left\lVert#1\right\rVert}
\newcommand{\transpose}{\intercal}

\usepackage{hyperref}
\author{Jim Fowler}
\course{Math 5520H}
\title{Normal operators}
\date{Tuesday, September 24, 2019}
\newcommand{\C}{\mathbb{C}}
\DeclareMathOperator{\image}{im}
\newcommand{\norm}[1]{\left\|#1\right\|}
\begin{document}
\maketitle

As usual, $V$ is a finite dimensional inner product space over the
complex numbers.

\begin{problem}
  What does it mean to say an operator $f : V \to V$ is \textbf{normal}?
\end{problem}

\vfill

\begin{problem}
  Suppose there is an orthonormal basis $\mathcal{B} = (\vec{e}_1, \ldots, \vec{e}_n)$ so that $[f]_{\mathcal{B}}$ is diagonal.  Show that $f$ is normal.
\end{problem}

\vfill

\begin{problem}
  There are normal operators for which the eigenvalues are not real.
\end{problem}

\vfill

\begin{problem}
  If $f$ is normal, then $\norm{f(\vec{v})} = \norm{f^\star(\vec{v})}$.
\end{problem}

\vfill

\begin{problem}
  If $f$ is normal, $\ker f = \ker f^\star$.
\end{problem}

\vfill

\begin{problem}
  If $f$ is normal, then $\ker f = (\image f)^\perp$.
%  The kernel of a normal operator is the orthogonal complement of its range.
\end{problem}

\vfill

\begin{problem}
  Eigenvectors of $f$ corresponding to distinct eigenvalues are orthogonal.
\end{problem}

\vfill

\begin{problem}
  If $f$ is normal, $\image f = \image f^\star$.
\end{problem}

\vfill

\begin{problem}
  By Schur, there is an orthonormal basis $\mathcal{B} = (\vec{e}_1, \ldots, \vec{e}_n)$ so that $[f]_{\mathcal{B}} = \left( f_{ij} \right)$ is upper triangular.  Compare $\norm{f(\vec{e}_1)}$ and $\norm{f^\star(\vec{e}_1)}$ to deduce that many of these entries are zero.
\end{problem}

\vfill

\begin{problem}
  Let $\vec{v} \in \ker f^{N+1}$ and compute
  $\langle f^\star(f^N(\vec{v})), f^{N-1}(\vec{v}) \rangle$ to conclude $\vec{v} \in \ker f^N$.
\end{problem}

\vfill

\begin{problem}
  If $f$ is normal, then $\ker f = \ker f^N$.
\end{problem}

\vfill

\begin{problem}
  What can be said about the generalized eigenvectors of a normal operator?
\end{problem}

\vfill

\end{document}

\begin{problem}
  The operator $f$ is self-adjoint iff for all $\vec{v} \in V$ we have $\langle f(\vec{v}),\vec{v} \rangle \in \mathbb{R}$.
\end{problem}

\vfill

\begin{problem}
  If for all $\vec{v} \in V$ we have $\langle f(\vec{v}, \vec{v} \rangle = 0$, then $f = 0$.
\end{problem}

\vfill 



%%% Local Variables:
%%% mode: latex
%%% TeX-master: t
%%% End:
