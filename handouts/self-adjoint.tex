\documentclass{homework}
\usepackage{amsmath}
\DeclareMathOperator{\Mat}{Mat}
\DeclareMathOperator{\End}{End}
\DeclareMathOperator{\Hom}{Hom}
\DeclareMathOperator{\id}{id}
\DeclareMathOperator{\image}{im}
\DeclareMathOperator{\Imag}{Imag}
\DeclareMathOperator{\rank}{rank}
\DeclareMathOperator{\nullity}{nullity}
\DeclareMathOperator{\trace}{tr}
\DeclareMathOperator{\Spec}{Spec}
\DeclareMathOperator{\Sym}{Sym}
\DeclareMathOperator{\pf}{pf}
\DeclareMathOperator{\Ortho}{O}

\newcommand{\C}{\mathbb{C}}

\DeclareMathOperator{\sla}{\mathfrak{sl}}
\newcommand{\norm}[1]{\left\lVert#1\right\rVert}
\newcommand{\transpose}{\intercal}

\usepackage{hyperref}
\author{Jim Fowler}
\course{Math 5520H}
\title{Self-adjoint operators}
\date{Tuesday, September 24, 2019}
\newcommand{\C}{\mathbb{C}}
\DeclareMathOperator{\image}{im}
\begin{document}
\maketitle

A \textbf{spectral theorem} is a theorem providing conditions for when
a linear operator can be diagonalized.

In what follows, $V$ is a finite dimensional inner product space over
the complex numbers and $f$ is a self-adjoint linear operator on $V$.

\begin{problem}
  What does it mean to say that $f : V \to V$ is self-adjoint?
\end{problem}

\vfill

\begin{problem}
  Suppose $\lambda$ is an eigenvalue of $f$.  Is ${E_\lambda}^\perp$ an invariant subspace of $f$?
\end{problem}

\vfill

\begin{problem}
  The operator $f$ can be diagonalized.
%  There is an orthonormal basis of $V$ consisting of eigenvectors of
%  $f$ for which the corresponding eigenvalues are real.
\end{problem}

\vfill

\begin{problem}
  All eigenvalues of $f$ are real.
\end{problem}

\vfill

\begin{problem}
  There is an orthonormal basis for $V$ consisting of eigenvectors of $f$.
\end{problem}

\vfill

\begin{problem}
  Formulate a related theorem for symmetric matrices with
  \textit{real} entries.  (\textit{Warning:} How do you know there is an eigenvector?)
\end{problem}

\vfill

\end{document}

\begin{problem}
  The operator $f$ is self-adjoint iff for all $\vec{v} \in V$ we have $\langle f(\vec{v}),\vec{v} \rangle \in \mathbb{R}$.
\end{problem}

\begin{problem}
  If for all $\vec{v} \in V$ we have $\langle f(\vec{v}, \vec{v} \rangle = 0$, then $f = 0$.
\end{problem}


