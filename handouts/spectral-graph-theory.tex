\documentclass{homework}
\usepackage{amsmath}
\DeclareMathOperator{\Mat}{Mat}
\DeclareMathOperator{\End}{End}
\DeclareMathOperator{\Hom}{Hom}
\DeclareMathOperator{\id}{id}
\DeclareMathOperator{\image}{im}
\DeclareMathOperator{\Imag}{Imag}
\DeclareMathOperator{\rank}{rank}
\DeclareMathOperator{\nullity}{nullity}
\DeclareMathOperator{\trace}{tr}
\DeclareMathOperator{\Spec}{Spec}
\DeclareMathOperator{\Sym}{Sym}
\DeclareMathOperator{\pf}{pf}
\DeclareMathOperator{\Ortho}{O}

\newcommand{\C}{\mathbb{C}}

\DeclareMathOperator{\sla}{\mathfrak{sl}}
\newcommand{\norm}[1]{\left\lVert#1\right\rVert}
\newcommand{\transpose}{\intercal}

\usepackage{hyperref}
\author{Jim Fowler}
\course{Math 5520H}
\title{Spectral graph theory}
\date{November 15, 2019}
\begin{document}
\maketitle

\begin{problem}
  What is a \textbf{graph}?
\end{problem}

\vfill

\begin{problem}
  For a graph $G$, what is the \textbf{adjacency matrix}, denoted $A_G$?
\end{problem}

\vfill

\begin{problem}
  What does the $ij$-entry of ${A_G}^k$ represent?
\end{problem}

\vfill

\begin{problem}
  For a graph $G$, what is the \textbf{Laplacian} matrix $L_G$?
\end{problem}

\vfill

\begin{problem}
  Why are the eigenvalues of $L_G$ real?
\end{problem}

\vfill

\begin{problem}
  Let $G(e)$ be the subgraph of $G$ consisting solely of edge $e$.  Identify $\sum_{e \in E(G)} L_{G(e)}$.
\end{problem}

\vfill

\begin{problem}
  For the edge $e = \{i,j\}$ in the graph $G$, compute
  $\langle v, L_{G(e)} v \rangle$.
\end{problem}

\vfill

\begin{problem}
  Show that $\langle v, L_{G(e)} v \rangle \geq 0$.  (In this case, we say that the symmetric operator $L_G$ is \textbf{positive semi-definite}.  If the inequality is strict, we say it is \textbf{positive definite}.)
\end{problem}

\vfill

\begin{problem}
  For a matrix $M$, is $M^\star M$ positive (semi-)definite?  
\end{problem}

\vfill

\begin{problem}
  What can be said about the spectrum of a positive semi-definite operator?
\end{problem}

\vfill

In the following, let
$\lambda_1 \leq \lambda_2 \leq \ldots \leq \lambda_n$ be the
eigenvalues of $L_G$.

\begin{problem}
  Compute $\lambda_1$.
\end{problem}

\vfill

\begin{problem}
  Suppose $G$ is connected.  Show $\lambda_2 > 0$.
\end{problem}

\vfill

\begin{problem}
  What feature of $G$ does $\dim \ker L_G$ describe?
\end{problem}

\vfill

\begin{problem}
  Show that $\lambda_n$ is the maximum of $\langle v, L_g v \rangle / \langle v, v \rangle$ for $v \neq 0$.  (This is a \textbf{min-max theorem}.)
\end{problem}

\vfill

\begin{problem}
  Pick an operator $f : \R^2 \to \R^2$ which is not self-adjoint but which has only real eigenvalues.  Show that $\langle v, f v \rangle / \langle v, v \rangle$ may exceed the largest eigenvalue of $f$.
\end{problem}

\vfill

\begin{problem}
  If there is a vertex of $G$ with degree $d$, then $\lambda_n \geq d$.
\end{problem}

\vfill

\begin{problem}
  If there is a vertex of $G$ with degree $d$, then $\lambda_n \geq d + 1$.
\end{problem}

\vfill

\end{document}
