\documentclass{homework}
\course{Math 5520H}
\author{Jim Fowler}
\usepackage{tikz-cd}
\usepackage{amsmath}
\DeclareMathOperator{\Mat}{Mat}
\DeclareMathOperator{\End}{End}
\DeclareMathOperator{\Hom}{Hom}
\DeclareMathOperator{\id}{id}
\DeclareMathOperator{\image}{im}
\DeclareMathOperator{\Imag}{Imag}
\DeclareMathOperator{\rank}{rank}
\DeclareMathOperator{\nullity}{nullity}
\DeclareMathOperator{\trace}{tr}
\DeclareMathOperator{\Spec}{Spec}
\DeclareMathOperator{\Sym}{Sym}
\DeclareMathOperator{\pf}{pf}
\DeclareMathOperator{\Ortho}{O}

\newcommand{\C}{\mathbb{C}}

\DeclareMathOperator{\sla}{\mathfrak{sl}}
\newcommand{\norm}[1]{\left\lVert#1\right\rVert}
\newcommand{\transpose}{\intercal}


\begin{document}
\maketitle

\begin{inspiration}
  I can assure you, at any rate, that my intentions are honourable and
  my results invariant, probably canonical, perhaps even
  functorial. But please allow me to assume that the characteristic is
  not two.\byline{Andr\'e Weil (as ``Rudolf Lipschitz'')}
\end{inspiration}

\section{Terminology}

\begin{problem}
  What is a \textit{quotient} of vector spaces?
\end{problem}

\begin{problem}
  What is the dual of $V$?  \textit{Warning: this problem set will
    challenge you to distinguish between adjoints of operators
    (\ref{definition-adjoint}) and duals of spaces, which are both
    denoted by a superscript $\star$.}
\end{problem}

\section{Numericals}

\begin{problem}
  The space of polynomials $\mathcal{P}_n$ has a basis
  $\{1,x,x^2,\ldots,x^n\}$ of monomials.  Find a dual basis, described
  via the derivative operator and evaluation functionals.
\end{problem}

\begin{problem}\label{f2-isomorphism-count}
  Let $V = \left(\mathbb{F}_2\right)^2$ and find an isomorphism
  $V \cong V^\star$.  How many such isomorphisms are there?
\end{problem}

\begin{problem}
  Suppose $M$ is the matrix
  \[
    M = \begin{pmatrix} 1 & 2 \\
      3 & 4 \\
      5 & 6 \\
      \vdots & \vdots \\
      99 & 100
      \end{pmatrix}
    \]
    Find the spectrum of the $50$-by-$50$ matrix $AA^\star$.  \textit{Hint:} The spectrum of $AA^\star$ is related to the spectrum of $A^\star A$, but how?
  \end{problem}


%when is the graph of a function a subspace?

%Hom in and out of a product (infinite, sum)

%dim R^oo / polynomials  = ?   show dual of polynomials is R^oo

%reprise homological algebra


\section{Exploration}

\begin{problem}
  Describe a ``natural'' map $D_V : V \to V^{\star\star}$ and verify
  that it is injective.  Under what circumstances is your map $D_V$ an
  isomorphism?
\end{problem}

\begin{problem}
  Explain how the \textbf{second derivative test from multivariable
    calculus} is related to the (real) spectral theorem.  Specifically, recall that when analyzing a critical point of $F : \R^n \to \R^n$, a key step was to build the \textbf{Hessian}, namely a matrix $H$ with entries
  \[
    H_{i,j} = \frac{\partial^2 F}{\partial x_j \partial x_j}.
  \]
  Is it true that $H_{i,j} = H_{j,i}$?  In that case, the Hessian is
  symmetric.  How does this help in deciding whether a critical point
  is a local maximum or local minimum or saddle?
\end{problem}

\begin{problem}\label{principal-component-analysis}An important application of the spectral theorem is
  \textbf{principal component analysis} or ``PCA.''

  Suppose $M$ is an $n$-by-$m$ matrix.  This is a \textbf{data matrix}
  so we think of each row as representing a different experiment and
  each column as representing a different sensor; write $\vec{v}_i$
  for column $i$.

  Suppose that the sensors are ``mean centered'' meaning that the
  averages of the columns are zero, or in symbols that if
  $1 \leq i \neq n$ then
  \(\displaystyle\frac{1}{m} \sum_{j=1}^m \left(\vec{v}_i\right)_j =
  0\).  The sensors, however, are likely to be correlated with each
  other.

  Explain why the dot product $\vec{v}_i \cdot \vec{v}_j$ reflects the
  correlation between sensor $i$ and $j$.  Is being ``mean centered''
  important?  Build a matrix $C$ with
  $C_{ij} = \vec{v}_i \cdot \vec{v}_j$ and explain how applying the
  spectral theorem to $C$ might reveal the ``structure'' of $M$.
\end{problem}

\begin{problem}\label{spectral-theorem-via-maxima}Suppose $V$ is a finite dimensional inner product space over $\C$,
  and suppose $M : V \to V$ is Hermitian.  Define
  \[
    S = \{ v \in V : |v| = 1 \},
  \]
  and a map $f : S \to \R$ via $f(v) = \langle Tv, v \rangle$.
  Suppose the maximum of $f$ is achieved at $w \in S$.  Show that $w$
  an eigenvector with eigenvalue $f(w)$.

  Does this argument also work over $\R$?  Did we invoke the
  Fundamental Theorem of Algebra somewhere?  How did we show that
  every operator on a finite dimensional vector space over $\C$ had an
  eigenvalue?
  
  Without an inner product, we relied on the Fundamental Theorem of
  Algebra to show that every operator on a finite dimensional vector
  space over $\C$ had an eigenvalue.
\end{problem}

% \begin{solution}
%   We first show that $w$ is an eigenvector with eigenvalue $f(w)$.
%   You can solve this problem quickly via Lagrange multipliers, but if
%   you would prefer not thinking about a constrained optimization
%   problem, extend $f$ to $f : V \to \R$ by the rule
%   $f(v) = \langle Tv, v \rangle / \langle v, v \rangle$.  Then let $v$
%   be a unit vector in $V$ with $\langle v, w \rangle = 0$.  For the
%   unit vector $w$ and any $\epsilon$ we have
%   $f(w) \geq f(w + \epsilon v)$, and since $f(w)$ is the maximum
%   value, we find the derivative ``in the tangent direction'' $v$ must
%   vanish, or in other words that $f(w + \epsilon v)$ differs from
%   $f(w)$ by something of order $\epsilon^2$.  To make this precise,
%   let's compute!
%   \begin{align*}
%     \langle Tw, w \rangle &\geq \frac{\langle T(w + \epsilon v), w + \epsilon v \rangle}{\langle w + \epsilon v, w + \epsilon v \rangle} \\
%                           &\geq \frac{1}{1 + |\epsilon|^2} \langle T(w + \epsilon v), w + \epsilon v \rangle,
%   \end{align*}
%   and after multiplying by $1 + |\epsilon|^2$ we find
%   \begin{align*}
%    |\epsilon|^2 \langle Tw, w \rangle &\geq \overline{\epsilon} \langle Tw, v \rangle + \epsilon \langle Tv, w \rangle + |\epsilon|^2 \langle Tv, v \rangle.
%   \end{align*}
%   Using the fact that $T$ is self-adjoint implies
%   \begin{align*}
%     |\epsilon|^2 \left( \langle Tw, w \rangle - \langle Tv, v \rangle \right) &\geq \overline{\epsilon \langle Tv, w \rangle} + \epsilon \langle Tv, w \rangle \\
%     &= \mbox{real part of } 2 \epsilon \langle Tv, w \rangle.
%   \end{align*}
%   Since $\langle Tw, w \rangle - \langle Tv, v \rangle \geq 0$, this means that 
%   \begin{align*}
%     |\epsilon|^2 \left( \mbox{something positive} \right) &\geq \mbox{real part of } 2 \epsilon \langle Tv, w \rangle,
%   \end{align*}
%   By replacing $\epsilon$ with $\pm \epsilon$ and $\pm i \epsilon$ and dividing by a positive real $\epsilon > 0$, we find
%   \begin{align*}
%     \epsilon \left( \mbox{positive} \right) &\geq \left| \mbox{real or imaginary of part of } 2 \langle Tv, w \rangle \right|,
%   \end{align*}
%   which implies $\langle Tw, v \rangle = 0$, and since $Tw$ is
%   orthogonal to all $v$ orthogonal to $w$, it must be that $Tw$ is a
%   multiple of $w$.
  
%   This arguments also works over $\R$ (where it is maybe even easier)
%   and does not depend on the Fundamental Theorem of Algebra.  Recall
%   that we showed an operator on a complex vector space had an
%   eigenvalue by invoking the Fundamental Theorem of Algebra, but that
%   tool is not needed to show this fact for self-adjoint operators!
% \end{solution}

\section{Prove or Disprove and Salvage if Possible}

\begin{problem}%salvage to $M$ normal
  Operators $M^\star$ and $M$ have conjugate eigenvalues and the same eigenvectors.
\end{problem}

\begin{problem}
  A normal operator $M$ is self-adjoint if and only if $\Spec M \subset \R$.
\end{problem}

\begin{problem}
  The composition of normal operators is normal.
\end{problem}

\begin{problem}% very much not unique!
  If $A$ is normal, then for every $n$ there is a unique matrix $B$ so that $B^n = A$.  (Compare \ref{polynomial-functional-calculus}.)
\end{problem}

\begin{problem}%this is so wrong but I think maybe a common
               %misconception?  salvage by noting iM is Hermitian, so
               %you can apply a spectral theorem
  If $M$ is skew-Hermitian (meaning $M^\star = -M$), then $M$ is
  normal, so if $\lambda$ is an eigenvalue of $M$, then
  $\bar{\lambda}$ is an eigenvalue of $M$.
\end{problem}

\begin{problem}\label{natural-transformation-to-double-dual}Suppose $f : V \to W$ is a linear map.  Then $f^{\star\star} : V^{\star\star} \to W^{\star\star}$ is also a linear map, and the diagram
  \[\begin{tikzcd}
      V \arrow{r}{f} \arrow{d}{D_V} & W \arrow{d}{D_W} \\
      V^{\star\star} \arrow{r}{f^{\star\star}}  & W^{\star\star}
    \end{tikzcd}\] commutes, meaning that
  $D_W \circ f = f^{\star\star} \circ D_V$.  In this case, we say
  ``there is a \textbf{natural transformation} from the identity
  functor to the double dual functor.''
\end{problem}


\end{document}

%%% Local Variables:
%%% mode: latex
%%% TeX-master: t
%%% End:
