\documentclass{homework}
\course{Math 5520H}
\author{Jim Fowler}

\DeclareMathOperator{\Ai}{Ai}
\DeclareMathOperator{\Bi}{Bi}

\begin{document}
\maketitle

\begin{inspiration}
They discover by this kind of experience that what they really wanted is usually not some collection of ``answers'' --- what they want is understanding.
\byline{William Thurston, \textit{On proof and progress in mathematics}}
\end{inspiration}

\section{Terminology}

\begin{problem}
  What is a positive operator?
\end{problem}

\begin{problem}
  What is a Sturm-Liouville problem?  A \textit{regular} Sturm-Liouville problem?
\end{problem}

\begin{problem}
  What is a Green's function?
\end{problem}

%http://www.math.jhu.edu/~lindblad/632/riesz.pdf

%https://en.wikipedia.org/wiki/Rayleigh_quotient#Use_in_Sturm–Liouville_theory



%https://en.wikipedia.org/wiki/Rayleigh_quotient

\section{Numericals}

\begin{problem}
  Multiply by $1/x$ to put $x^2 y'' + xy' + \lambda y = 0$  into Sturm-Liouville form.
\end{problem}

\begin{problem}\label{wave-with-fixed-endpoints}Find the eigenfunctions and eigenvalues of the boundary value problem
  \begin{align*}
    y(x)'' &= - \lambda y(x) \mbox{ for $x \in [a,b]$,} \\
    y(a) &= y(b) = 0.
  \end{align*}
  Is there a smallest eigenvalue?  A largest eigenvalue?  How many
  eigenvalues are there?
\end{problem}

\begin{solution}
  That there are infinitely many eigenvalues, bounded below, is a
  general theorem about SL problems.
\end{solution}

\section{Exploration}

\begin{problem}\label{raleigh-quotients}Let $M : V \to V$ be a Hermitian operator on an $n$-dimensional
  space, with eigenvalues $\lambda_i \in \R$ arranged in order so that
  $y_1 \leq y_2 \leq \cdots \leq \lambda _n$.

  For $x \in V$, define the \textbf{Rayleigh quotient} by
  \[R(M,x)=\frac{\langle x, Mx \rangle}{\langle x, x \rangle}.\]
  Invoke the Spectral Theorem to show that
  $\lambda_1 \leq R(M,x) \leq \lambda_n$.  (This extends
  \ref{spectral-theorem-via-maxima}.)
\end{problem}

\begin{problem}\label{pi-squared-less-than-ten}Multiply $Ly = -\lambda y$ by $y$ and integrate to find
  \[
    \lambda = \frac{ -\displaystyle\int_a^b \left( y (p y')' - q y^2  \right) \, dx }{\displaystyle\int_a^b y^2 \, dx},
  \]
  which, after integrating by parts, yields the Rayleigh quotient
  \[
    R(L,y) =  {\frac {\left(\left.-p(x)y(x)y'(x)\right|_{a}^{b}\right)+\left(\displaystyle\int _{a}^{b}\left(p(x)y'(x)^{2}-q(x)y(x)^{2}\right)\,dx\right)}{\displaystyle\int _{a}^{b}{y(x)^{2}}\,dx}}.
  \]
  By \ref{raleigh-quotients}, $R(L,y)$ overestimates the smallest eigenvalue.

  Solve \ref{wave-with-fixed-endpoints} with $a = 0$ and $b = 1$, meaning $y'' = -\lambda y$ subject to $y(0) = y(1) = 0$.  What is the smallest eigenvalue $\lambda_1$ in this case?

  Picking $y = x - x^2$, we find $y$ is not an eigenfunction, but in this case
  \[
   R(L,y) = \frac{\displaystyle\int_0^1 y'(x)^2 \, dx}{\displaystyle\int _{0}^{1}{y(x)^{2}}\,dx} = 10.
 \]
 Is it the case that $10 \geq \lambda_1$?  (I find this fact helpful in mid-March each year.)
\end{problem}

\begin{problem}
  Use the method of \ref{pi-squared-less-than-ten} to show that if
  \[
    \left( (1+x^{2}) y' \right)' =  -\lambda y \mbox{ and } y(0) = y(1) = 0,
  \]
  then $\lambda \geq 0$.  Can you also show that $\lambda \neq 0$?
\end{problem}

\begin{solution}
  In this case, $Ly = ((1+x^{2})y')'$.   For a function $y$ satisfying the boundary conditions but not necessary an eigenfunction, we compute
    \[
      R(L,y) =  \frac{\displaystyle\int _{0}^{1} (1+x^{2})y'(x)^{2} \,dx}
    {\displaystyle\int _{a}^{b}{y(x)^{2}}\,dx}
\]
and note that $R(L,y) \geq 0$ because the integrands are nonnegative.
\end{solution}

\begin{problem}
  Recall the Airy functions (\ref{introduction-airy-function}) which
  we'll denote $\Ai$ and $\Bi$.  Note that $y(x) = \Ai(x - \lambda)$
  and $y(x) = \Bi(x-\lambda)$ are solutions to
  $y'' + (\lambda - x) y = 0$.

  For certain $\lambda$, it is possible to find $\alpha$ and $\beta$ so that
  \[
    y(x) = \alpha \Ai(x - \lambda) + \beta \Bi(x - \lambda)
  \]
  is a solution to $y'' + (\lambda - x) y = 0$ and satisfies
  $y(0) = y(1) = 0$.  Note that the boundary conditions are the
  challenge here!  Find a reasonably good upper bound on the smallest
  such $\lambda$.

  You can check your estimate with the SageMath command
\begin{verbatim}
find_root( lambda x: airy_ai(1-x)*airy_bi( -x)-
                     airy_ai( -x)*airy_bi(1-x), 
           0, 20)
\end{verbatim}
\end{problem}


\section{Prove or Disprove and Salvage if Possible}
%http://www.math.iitb.ac.in/~siva/ma41707/ode7.pdf

\begin{problem}
  If $\lambda$ is an eigenvalue of a regular Sturm-Lioville boundary
  value problem, then $\lambda \in \R$.
\end{problem}

\begin{problem}% FALSE: this is the wrong inner product
  With respect to the inner product
    \[
      \langle f, g \rangle = \displaystyle\int_a^b f(x) \, \overline{g(x)} \, dx,
    \]
    eigenfunctions of a regular Sturm-Liouville problem on $[a,b]$
    corresponding to different eigenvalues are orthogonal.
\end{problem}

\begin{problem}
  Let $y_1$ and $y_2$ be two eigenfunctions of a regular
  Sturm-Lioville boundary value problem.  If $y_1$ and $y_2$
  correspond to the same eigenvalue, then $y_1$ is a multiple of
  $y_2$.
\end{problem}


\end{document}
