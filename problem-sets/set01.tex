\documentclass{homework}
\course{Math 5520H}
\author{Jim Fowler}
\usepackage{amsmath}
\DeclareMathOperator{\Mat}{Mat}
\DeclareMathOperator{\End}{End}
\DeclareMathOperator{\Hom}{Hom}
\DeclareMathOperator{\id}{id}
\DeclareMathOperator{\image}{im}
\DeclareMathOperator{\Imag}{Imag}
\DeclareMathOperator{\rank}{rank}
\DeclareMathOperator{\nullity}{nullity}
\DeclareMathOperator{\trace}{tr}
\DeclareMathOperator{\Spec}{Spec}
\DeclareMathOperator{\Sym}{Sym}
\DeclareMathOperator{\pf}{pf}
\DeclareMathOperator{\Ortho}{O}

\newcommand{\C}{\mathbb{C}}

\DeclareMathOperator{\sla}{\mathfrak{sl}}
\newcommand{\norm}[1]{\left\lVert#1\right\rVert}
\newcommand{\transpose}{\intercal}


\begin{document}
\maketitle

\begin{inspiration}
In mathematics, existence is freedom from contradiction.
\byline{David Hilbert}
\end{inspiration}

\section{Terminology}

\begin{problem}
  What are elementary row operations?  What does it mean to say that
  two matrices are ``row equivalent'' matrices?
\end{problem}

\begin{problem}
  What is a homogeneous system of linear equations?
\end{problem}

\begin{problem}
  What is an augmented matrix?
\end{problem}

\begin{problem}
  What does it mean to say that a matrix is in ``row echelon'' form?
\end{problem}

\begin{problem}
  What is the rank of a matrix?
\end{problem}

\section{Numericals}

\begin{problem}
  Set $v = (1,2,3)$ and $w = (3,2,1)$ and $u = (1,1,1)$.  Find real
  numbers $\alpha$ and $\beta$ so that $\alpha v + \beta w = u$.
\end{problem}

\begin{problem}
  Find all solutions to the following system of linear equations:
  \begin{align*}
    \phantom{}x+y\phantom{+z+w} &= 1, \\
    \phantom{x+}y+z\phantom{+w} &= 2, \\
    \phantom{x+y+}z+w\phantom{} &= 3, \\
    x+\phantom{y+z+}w\phantom{} &= 4. \\    
  \end{align*}
\end{problem}

\begin{problem}
  Find all $\alpha, \beta \in \R$ so that for all $x \in \R$ we have $\alpha \sin x + \beta \cos x = 0$.  \textit{Hint:}~Relate this to solving a linear system of equations.
\end{problem}

\begin{problem}
  Find all $(x,y,z) \in \R^3$ so that $\alpha^3 x + \alpha^2 y + \alpha z = 0$ for $\alpha \in \{1,2\}$.  What if we instead demand the equation holds for $\alpha \in \{1,2,3\}$?
\end{problem}

\begin{problem}
  Pick real numbers $x$, $y$, $z$ and row reduce the matrix
  \[
    \begin{pmatrix}
      x^2 & y^2 & z^2 \\
      x^1 & y^1 & z^1 \\
      x^0 & y^0 & z^0            
    \end{pmatrix}.
  \]
\end{problem}

\section{Exploration}

\begin{problem}
  Suppose $M$ can be transformed into $N$ by applying elementary row
  operations.  Can you likewise transform $N$ into $M$?
\end{problem}

\begin{problem}
  A quadrilateral in the plane $\R^2$ has vertices $A$, $B$, $C$, and
  $D$.  Let $M_{\overline{AB}}$ be the midpoint of side
  $\overline{AB}$, and similarly define $M_{\overline{BC}}$ and
  $M_{\overline{CD}}$ and $M_{\overline{DA}}$.  Consider the
  quadrilateral having these four midpoints as its vertices; what sort
  of polygon is it?
\end{problem}

\begin{problem}
  Write down a randomly chosen $2$-by-$2$ matrix $M$, and apply row
  operations to find a matrix in row echelon form which is row
  equivalent to $M$.  What usually happens?
\end{problem}

\begin{problem}
  When are two $2$-by-$2$ matrices row equivalent?
\end{problem}

\section{Prove or Disprove and Salvage if Possible}

\begin{problem}
  The homogeneous system of linear equations $\mathcal{L}(M)$
  associated with the matrix $M$ has a solution set $S$.  If $N$ is
  another matrix row equivalent to $M$, then the solution set to
  $\mathcal{L}(N)$ is also $S$.
\end{problem}

\begin{problem}
  Let $M$ and $N$ be matrices which both have $c$ columns, and further
  let $S$ and $T$ be the solution set associated with $\mathcal{L}(M)$
  and $\mathcal{L}(N)$, respectively.  Then the homogeneous system of
  linear equations associated to the block matrix $\begin{pmatrix} M \\
    N \end{pmatrix}$ has solution set $S \cup T$.
\end{problem}

\begin{problem}
  There exists a nonzero matrix which is row equivalent to the zero matrix.
\end{problem}

\begin{problem}
  Suppose two $m$-by-$n$ matrices $A$ and $B$ have the same rank.
  Then $A$ and $B$ are row equivalent.
\end{problem}

\begin{problem}
  There exists a non-identity matrix which is row equivalent to the identity
  matrix.
\end{problem}

\begin{problem}
  There exists a symmetric matrix which is row equivalent to a matrix which
  is not symmetric.
\end{problem}

\begin{problem}
  A $3$-by-$3$ matrix is the sum of three rank $1$ matrices.
\end{problem}

\begin{problem}
  An $n$-by-$n$ matrix with all entries equal to 1 has rank $n$.
\end{problem}

\begin{problem}
  The sum of rank $n$ matrices has rank $n$.
\end{problem}

\begin{problem}
  Every diagonal matrix is a multiple of the identity matrix.
\end{problem}

\end{document}
