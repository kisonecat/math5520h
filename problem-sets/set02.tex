\documentclass{homework}
\course{Math 5520H}
\author{Jim Fowler}
\usepackage{amsmath}
\DeclareMathOperator{\Mat}{Mat}
\DeclareMathOperator{\End}{End}
\DeclareMathOperator{\Hom}{Hom}
\DeclareMathOperator{\id}{id}
\DeclareMathOperator{\image}{im}
\DeclareMathOperator{\Imag}{Imag}
\DeclareMathOperator{\rank}{rank}
\DeclareMathOperator{\nullity}{nullity}
\DeclareMathOperator{\trace}{tr}
\DeclareMathOperator{\Spec}{Spec}
\DeclareMathOperator{\Sym}{Sym}
\DeclareMathOperator{\pf}{pf}
\DeclareMathOperator{\Ortho}{O}

\newcommand{\C}{\mathbb{C}}

\DeclareMathOperator{\sla}{\mathfrak{sl}}
\newcommand{\norm}[1]{\left\lVert#1\right\rVert}
\newcommand{\transpose}{\intercal}


\begin{document}
\maketitle

\begin{inspiration}
That's all well and good in practice, but how does it work in theory?
\byline{Shmuel Weinberger}
\end{inspiration}

\section{Terminology}

\begin{problem}
  What is a field?
\end{problem}

\begin{problem}
  What is a vector space?
\end{problem}

\begin{problem}
  What is a subspace?
\end{problem}

\begin{problem}
  What is a basis?  What is an ordered basis?
\end{problem}

\begin{problem} % check students are okay handling indices; also want to point towards adjoint
  What is the transpose of a matrix?
\end{problem}

\section{Numericals}

%check if certain subsets of continuous functions are subspaces?

\begin{problem}
  Show that $\{ (2,1), (1,1) \}$ is a basis for $\R^2$.
\end{problem}

\begin{problem}
  Let $\mathcal{P}_{3}(\R)$ be polynomials (over $\R$) of degree $\leq 3$.  Verify that
  \[ \left\{ \binom{x}{0}, \binom{x}{1}, \binom{x}{2}, \binom{x}{3} \right\} \]
  spans $\mathcal{P}_{3}(\R)$.  This is \textbf{Newton interpolation.}
\end{problem}

\begin{problem}
  Find two polynomials $p,q \in \mathcal{P}_{2}(\R)$ so that $\{p,q\}$ are linearly independent, but $\{p',q'\}$ are linearly dependent.  Here $p'$ and $q'$ are the derivatives of $p$ and $q$, respectively.
\end{problem}

\begin{problem}
  Find a $2$-by-$2$ matrix $A$ so that $(A^n)_{11} = 2^n$.  \\
  Then find a $2$-by-$2$ matrix $B$ so that $(B^n)_{12} = n$.
\end{problem}

\begin{problem}
  Find a square matrix $M$ so that $M^4 = 0$ but $M^3 \neq 0$.
\end{problem}

\begin{problem}
  Recall from a calculus class that
  $\frac{1}{1-x} = 1 + x + x^2 + x^3 + \cdots$.  Bravely replace $x$
  by a certain matrix in order to invert
  $\begin{pmatrix} 1 & 2 & 3 \\ 0 & 1 & 2 \\ 0 & 0 & 1 \end{pmatrix}$.
\end{problem}

\section{Exploration}

\begin{problem}
  Given vector spaces $V$ and $W$, endow $V \times W$ with the
  structure of a vector space.  (We'll write $V \oplus W$ for this
  vector space.)
\end{problem}

\begin{problem}
  What is $\dim \Mat_{n \times m}(\R)$?
\end{problem}

\begin{problem}
  Let $S$ be the set of degree $\leq 3$ polynomials (with real
  coefficients) which vanish at the point $2$, i.e.,
  \[ S = \{ p \in \mathcal{P}_{3}(\R) : p(2) = 0 \}. \] Explain
  why $S$ is a subspace of the vector space of polynomials
  $\mathcal{P}(\R)$.  What is $\dim S$?
\end{problem}

\begin{problem}
  Suppose a subspace $V \subset \R^n$ is spanned by
  $\{v_1,\ldots,v_n\}$.  If you assemble these vectors together as the
  rows of matrix, how does the rank of that matrix
  (\ref{definition-rank}) relate to the dimension of $V$?  Does ``row
  reducing'' help you find a basis for $V$?
\end{problem}

\begin{problem}\label{frobenius-matrix}A matrix $M$ is a \textbf{Frobenius matrix} if $M_{ii} = 1$, all
  entries above the diagonal are zero, and with the exception of at
  most one column all entries below the diagonal are zero.  An example is
  \[\begin{pmatrix}
      1&0&0&\cdots &0\\
      0&1&0&\cdots &0\\
      0&a_{32}&1&\cdots &0\\
      \vdots &\vdots &\vdots &\ddots &\vdots \\
      0&a_{n2}&0&\cdots &1
    \end{pmatrix}.\]
  Show that the inverse of a Frobenius matrix is again Frobenius.
\end{problem}

\begin{problem}
  Find a function $f : \Mat_{2\times 2}(\R) \to \R$ so that, for all
  $2$-by-$2$ matrices $A$ and $B$, we have $f(AB) = f(BA)$.
\end{problem}

\section{Prove or Disprove and Salvage if Possible}

\begin{problem}
  The set of polynomials $\{ (1+x)^n : n \in \N \}$ is a basis for
  $\mathcal{P}(\R)$.
\end{problem}

\begin{problem}
  Eventually constant functions form a subspace of the vector space of
  all functions $\R \to \R$.  (A function $f : \R \to \R$ is eventually
  constant if there exists an $a,b \in \R$ so that for all $x \geq b$
  we have $f(x) = a$.)
\end{problem}

\begin{problem}\label{sin-cos-indie}The set $\{ \sin, \cos \}$ is a linearly independent subset of the
  vector space of $2\pi$-periodic square-integrable functions;
  cf.~\ref{sin-cos-linear-system}.
\end{problem}

\begin{problem}
  For a vector space $V$ and for finite sets $A, B \subset V$ with
  $A \subset B$, if $A$ is linearly independent then $B$ is linearly
  independent.
\end{problem}

\begin{problem}\label{steinitz-exchange-lemma}If the vector space $V$ is spanned by $S = \{v_1,\ldots,v_n\}$,
  and the set $B = \{b_1,\ldots,b_m \} \subset V$ is linearly
  independent, then $m \leq n$ and there is a subset $S' \subset S$
  with $\# S' = n-m$ such that the set $B \cup S'$ spans $V$.
\end{problem}

\begin{problem}
  Suppose $B$ and $B'$ are both bases for a finite dimensional vector
  space $V$.  Then $\# B = \# B'$.
\end{problem}

\begin{problem}
  Let $V = \{ (a_1,a_2,\ldots) : a_i \in \R \}$ be the vector space of
  sequences of real numbers.  Define $e_j \in V$ by the rule
  \[
    (e_j)_i = \delta_{ij} = \begin{cases}
      1 & \mbox{if $i = j$,} \\
      0 & \mbox{if $i \neq j$.}
    \end{cases}
  \]
  Then $\{ e_j : j \in \N \}$ spans $V$.
\end{problem}

\begin{problem}
  The inverse of an upper triangular matrix is upper triangular.
\end{problem}

\end{document}

%%% Local Variables:
%%% mode: latex
%%% TeX-master: t
%%% End:
