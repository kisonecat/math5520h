\documentclass{homework}
\usepackage{amsmath}
\DeclareMathOperator{\Mat}{Mat}
\DeclareMathOperator{\End}{End}
\DeclareMathOperator{\Hom}{Hom}
\DeclareMathOperator{\id}{id}
\DeclareMathOperator{\image}{im}
\DeclareMathOperator{\Imag}{Imag}
\DeclareMathOperator{\rank}{rank}
\DeclareMathOperator{\nullity}{nullity}
\DeclareMathOperator{\trace}{tr}
\DeclareMathOperator{\Spec}{Spec}
\DeclareMathOperator{\Sym}{Sym}
\DeclareMathOperator{\pf}{pf}
\DeclareMathOperator{\Ortho}{O}

\newcommand{\C}{\mathbb{C}}

\DeclareMathOperator{\sla}{\mathfrak{sl}}
\newcommand{\norm}[1]{\left\lVert#1\right\rVert}
\newcommand{\transpose}{\intercal}

\course{Math 5520H}
\author{Jim Fowler}

\begin{document}
\maketitle

\begin{inspiration}
Unfortunately, no one can be told what the Matrix is. You have to see it for yourself.
\byline{Morpheus in \textit{The Matrix}}
\end{inspiration}

\section{Terminology}

\begin{problem}
  What is  a linear transformation?  (Recall \ref{terminology:matrix}.)
\end{problem}

\begin{problem}
  What is the kernel of a linear transformation?  The image?  Rank?  Nullity?
\end{problem}

\begin{problem}\label{definition-inner-product}
  What is an inner product space?
\end{problem}

\begin{problem}\label{definition-commutator}
  What is the commutator $[A,B]$?
\end{problem}

\section{Numericals}

\begin{problem}
  Use the basis of binomial coefficients (from
  \ref{newton-interpolation}) to write down a matrix representing the
  derivative operator on $\mathcal{P}_{3}(\R)$.
\end{problem}

\begin{problem}
  Suppose $T : \R^5 \to \R^3$ is the linear transformation associated to the matrix
  \[\begin{bmatrix}
    1 & 0 & 1 & 1 & 0 \\
    0 & 1 & 2 & 4 & 8 \\
    0 & 0 & 0 & 0 & 0
  \end{bmatrix}\]
  with respect to the standard bases of $\R^5$ and $\R^3$.  Compute the rank and nullity of $T$.
\end{problem}

\begin{problem}
  Suppose $n \geq 2$ and $\mathcal{B} = (e_0,\ldots,e_{n-1})$ is an ordered basis of $V$.  Find the rank and nullity of the transformation $f : V \to V$ defined by $f(e_i) = e_i + e_{i+1 \mod n}$.
\end{problem}

\begin{problem}
  On the vector space of polynomials $\R[x]$, define two linear operators $M_x, \frac{d}{dx} : \R[x] \to \R[x]$ by the rules $M_x(f) = x \, f$ and $\frac{d}{dx}(f) = f'$, the derivative.  Do these operators commute?  Compute $[M_x,\frac{d}{dx}]$.
\end{problem}

\begin{problem}
  For a $2$-by-$2$ matrix $A$, define the transformation
  $f_A : \Mat_{2\times 2}(\R) \to \Mat_{2\times 2}(\R)$ by the rule
  $f_A(M) = AM$.  Choose an ordered basis
  $\mathcal{B}$ of $\Mat_{2\times 2}(\R)$ and compute
  $[f_A]_{\mathcal{B}}$.  What is the rank and nullity of $[f_A]_{\mathcal{B}}$?
\end{problem}

\section{Exploration}

\begin{problem}
  Mathematics is like a collection of analogies.  The ``null space''
  of a matrix is analogous to the kernel of a transformation; the
  ``column space'' (or is it the ``row space''?) of a matrix is
  related to the image of a corresponding transformation.  Explain these relationships.
\end{problem}

\begin{problem}
  For a vector space $V$, the space of linear transformations
  $\End(V) = \Hom(V,V)$ is a vector space.  Is it a ring (meaning in a
  way compatible with the vector space structure)?  A field?  An
  algebra over a field?
\end{problem}

\begin{problem}
  Given mysterious linear transformations $f : V \to W$ and
  $g : W \to U$ between finite dimensional vector spaces and knowing
  only the ranks of $f$ and $g$ and the dimensions of the spaces
  involved, how big or small could $\rank (g \circ f)$ be?
\end{problem}

\begin{problem}
  For a square matrix $M$, the \textbf{minimal polynomial} of $M$ is
  the monic polynomial $p$ of least degree such that $p(M) = 0$.
  (What does it mean to evaluate a polynomial at a matrix?)  What is
  the minimal polynomial of the transformation $f : V \to V$ given by
  $f(v) = 2v$?
\end{problem}

\section{Prove or Disprove and Salvage if Possible}

\begin{problem}
  The function $E_2 : \mathcal{P}(\R) \to \R$ given by $E_2(f) = f(2)$ is a linear transformation.
\end{problem}

\begin{problem}
  Suppose $V$ and $U$ are subspaces of a finite dimensional vector
  space, and let $W$ be the smallest subspace containing $V$ and $U$.
  Then $\dim W = \dim V + \dim U$.
\end{problem}

\begin{problem}
  A maximal linearly independent set is a basis.
\end{problem}

\begin{problem}
  Every vector space has a basis.
\end{problem}

\begin{problem}
  The kernel of the linear transformation $T : V \to W$ is a subspace of $V$.
\end{problem}

\begin{problem}
  The image of a basis is a basis.
\end{problem}

\begin{problem} % a couple possible salvages
  A linear transformation $T : V \to V$ is injective iff $\ker T = V$.
\end{problem}

\begin{problem}
  A linear transformation $T : V \to V$ is either surjective or has nontrivial kernel.
\end{problem}

\begin{problem}
  Suppose $T : V \to V$ is a linear transformation.  Then because of the rank-nullity theorem, $\ker T \cap \image T = \{0\}$.
\end{problem}

\begin{problem}
  Suppose $f, g: V \to V$ are linear transformations with the property that $f \circ g = \id$.  Then $g \circ f = \id$.
\end{problem}

\begin{problem}
  Suppose $f : V \to V$ has the property for all $v \in V$ that $(f \circ f)(v) = 0$.  Then $\ker f \subset \image f$.
\end{problem}

\begin{problem}
  Given linear operators $f, g : V \to V$ on a finite dimensional vector space $V$, suppose there is an ordered basis $\mathcal{B}$ of $V$ so that $[f]_{\mathcal{B}}$ and $[g]_{\mathcal{B}}$ are both diagonal matrices.  Then $f \circ g = g \circ f$.
\end{problem}

\begin{problem}
  Given linear operators $f, g : V \to V$ on a finite dimensional
  vector space $V$ with the property $f \circ g = g \circ f$, there is
  an ordered basis $\mathcal{B}$ of $V$ so that $[f]_{\mathcal{B}}$
  and $[g]_{\mathcal{B}}$ are both diagonal matrices.
\end{problem}

\begin{problem}
  The vector space $V \cong \R^2$ has ordered basis $\mathcal{B} = (e_1,e_2)$ and the linear map $f : V \to V$ is represented by
  \[
    [f]_{\mathcal{B}} = \begin{bmatrix} a & b \\ 0 & c \end{bmatrix}.
  \]
  Then the minimal polynomial of $f$ is $(x-a)(x-c)$.
\end{problem}

\end{document}
