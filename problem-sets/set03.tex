\documentclass{homework}
\usepackage{amsmath}
\DeclareMathOperator{\Mat}{Mat}
\DeclareMathOperator{\End}{End}
\DeclareMathOperator{\Hom}{Hom}
\DeclareMathOperator{\id}{id}
\DeclareMathOperator{\image}{im}
\DeclareMathOperator{\Imag}{Imag}
\DeclareMathOperator{\rank}{rank}
\DeclareMathOperator{\nullity}{nullity}
\DeclareMathOperator{\trace}{tr}
\DeclareMathOperator{\Spec}{Spec}
\DeclareMathOperator{\Sym}{Sym}
\DeclareMathOperator{\pf}{pf}
\DeclareMathOperator{\Ortho}{O}

\newcommand{\C}{\mathbb{C}}

\DeclareMathOperator{\sla}{\mathfrak{sl}}
\newcommand{\norm}[1]{\left\lVert#1\right\rVert}
\newcommand{\transpose}{\intercal}

\course{Math 5520H}
\author{Jim Fowler}

\begin{document}
\maketitle

\begin{inspiration}
Unfortunately, no one can be told what the Matrix is. You have to see it for yourself.
\byline{Morpheus in \textit{The Matrix}}
\end{inspiration}

\section{Terminology}

\begin{problem}
  What is  a linear transformation?  (Recall \ref{terminology:matrix}.)
\end{problem}

\begin{problem}
  What is the kernel of a linear transformation?  The image?  Rank?  Nullity?
\end{problem}

\begin{problem}\label{definition-inner-product}
  What is an inner product space?
\end{problem}

\begin{problem}\label{definition-commutator}
  What is the commutator $[A,B]$?
\end{problem}

\section{Numericals}

\begin{problem}
  Use the basis of binomial coefficients (from
  \ref{newton-interpolation}) to write down a matrix representing the
  derivative operator on $\mathcal{P}_{3}(\R)$.
\end{problem}

\begin{problem}
  Suppose $T : \R^5 \to \R^3$ is the linear transformation associated to the matrix
  \[\begin{bmatrix}
    1 & 0 & 1 & 1 & 0 \\
    0 & 1 & 2 & 4 & 8 \\
    0 & 0 & 0 & 0 & 0
  \end{bmatrix}\]
  with respect to the standard bases of $\R^5$ and $\R^3$.  Compute the rank and nullity of $T$.
\end{problem}

\begin{problem}
  Suppose $n \geq 2$ and $\mathcal{B} = (e_0,\ldots,e_{n-1})$ is an ordered basis of $V$.  Find the rank and nullity of the transformation $f : V \to V$ defined by $f(e_i) = e_i + e_{i+1 \mod n}$.
\end{problem}

\begin{problem}\label{derivative-multiplication-commute}On the vector space of polynomials $\R[x]$, define two linear operators $M_x, \frac{d}{dx} : \R[x] \to \R[x]$ by the rules $M_x(f) = x \, f$ and $\frac{d}{dx}(f) = f'$, the derivative.  Do these operators commute?  Compute $[M_x,\frac{d}{dx}]$.
\end{problem}

\begin{solution}
  These operators do not commute.  Let $p = \sum_{n=0}^\infty c_n x^n$
  with all but finitely many of the $c_n$ equal to zero.  Then
  $p' = \sum_{n=1}^\infty n c_n x^{n-1}$, and $M_x p' = \sum_{n=1}^\infty n c_n x^{n}$.  On the other hand, $M_x p = \sum_{n=0}^\infty c_n x^{n+1}$ and $\frac{d}{dx} M_x p = \sum_{n=0}^\infty (n+1) c_n x^n$. Therefore $[M_x,\frac{d}{dx}] p = \sum_{n=1}^\infty n c_n x^{n} - \sum_{n=0}^\infty (n+1) c_n x^n = -p$.  In other words, $[M_x,\frac{d}{dx}]$ is $-\id$.
\end{solution}

\begin{problem}
  For a $2$-by-$2$ matrix $A$, define the transformation
  $f_A : \Mat_{2\times 2}(\R) \to \Mat_{2\times 2}(\R)$ by the rule
  $f_A(M) = AM$.  Choose an ordered basis
  $\mathcal{B}$ of $\Mat_{2\times 2}(\R)$ and compute
  $[f_A]_{\mathcal{B}}$.  What is the rank and nullity of $[f_A]_{\mathcal{B}}$?
\end{problem}

\begin{solution}
  If $A = 0$, then $f_A$ is the zero map and $[f_A]_{\mathcal{B}}$ has
  rank zero and nulilty four.  If $A$ has rank two, then $A$ is
  invertible, and $f_{A^{-1}}$ is the inverse for $f_A$, so
  $[f_A]_{\mathcal{B}}$ has rank four and nullity zero.  The most
  interesting case to work out is when $A$ has rank $1$ in which case
  the rank and nullity of $[f_A]_{\mathcal{B}}$ is two.
\end{solution}

\section{Exploration}

\begin{problem}
  Mathematics is like a collection of analogies.  The ``null space''
  of a matrix is analogous to the kernel of a transformation; the
  ``column space'' (or is it the ``row space''?) of a matrix is
  related to the image of a corresponding transformation.  Explain these relationships.
\end{problem}

\begin{problem}
  For a vector space $V$, the space of linear transformations
  $\End(V) = \Hom(V,V)$ is a vector space.  Is it a ring (meaning in a
  way compatible with the vector space structure)?  A field?  An
  algebra over a field?
\end{problem}

\begin{solution}
  The space $\Hom(V)$ is an algebra over the same field over which $V$
  is defined.  The bilinear product is composition.  (If you like this
  sort of thing, you may enjoy reading about $C^\star$-algebras.)
\end{solution}

\begin{problem}
  Given mysterious linear transformations $f : V \to W$ and
  $g : W \to U$ between finite dimensional vector spaces and knowing
  only the ranks of $f$ and $g$ and the dimensions of the spaces
  involved, how big or small could $\rank (g \circ f)$ be?
\end{problem}

\begin{solution}
  Working this out in detail is a great exercise in the rank-nullity
  theorem!  It is possible that $\rank (g \circ f)$ could be as big as
  the minimum of the ranks of $f$ and $g$.  In the worst case, $g$
  kills as much of the image of $f$ as possible, in which case the
  $\rank (g \circ f)$ is as small as
  $\rank f - \nullity g = \rank f - \dim W + \rank g$.
\end{solution}

\begin{problem}\label{definition-minimal-polynomial}For a square matrix $M$, the \textbf{minimal polynomial} of $M$ is
  the monic polynomial $p$ of least degree such that $p(M) = 0$.
  (What does it mean to evaluate a polynomial at a matrix?)  What is
  the minimal polynomial of the transformation $f : V \to V$ given by
  $f(v) = 2v$?
\end{problem}

\begin{solution}
  The polynomial $p(x) = x - 2$ is such that $p(M) = 0$.
\end{solution}

\section{Prove or Disprove and Salvage if Possible}

\begin{problem}
  The function $E_2 : \mathcal{P}(\R) \to \R$ given by $E_2(f) = f(2)$ is a linear transformation.
\end{problem}

\begin{solution}
  This is linear.  Indeed, $E_2(f + g) = (f+g)(2) = f(2) + g(2) = E_2(f) + E_2(g)$ and likewise for scalar multiplication.
\end{solution}

\begin{problem}
  Suppose $V$ and $U$ are subspaces of a finite dimensional vector
  space, and let $W$ be the smallest subspace containing $V$ and $U$.
  Then $\dim W = \dim V + \dim U$.
\end{problem}

\begin{solution}
  This is true with the proviso that $V \cap U = \{0\}$, in which case
  we might even write $W = V \oplus U$.

  Suppose $\{ v_i \}$ and $\{u_i\}$ are bases for $V$ and $U$,
  respectively.  Then $\{v_i\} \cup \{u_i\}$ is a basis for $W$.  That
  this spans is easy (and does not depend on $V \cap U = \{0\}$).
  That this is linearly independent is harder to see: a linear
  dependence $\sum \lambda_i v_i + \sum \mu_j u_j = 0$ can be written
  as $\sum \lambda_i v_i =-\sum \mu_j u_j$, which by the assumption
  $V \cap U = \{0\}$ gives rise to linear dependences for $\{v_i\}$
  and $\{u_i\}$, making this a trivial dependence.
\end{solution}

\begin{problem}
  A maximal linearly independent set is a basis.
\end{problem}

\begin{solution}
  Suppose $S$ is a linearly independent subset of a vector space $V$
  with the property that, for any $v \in V \setminus S$, the set
  $S \cup \{v\}$ is linearly dependent.  The nontrivial linear
  dependence cannot involve $v$, so it the nontrivial linear
  dependence actually shows $v$ is in the span of $S$.  Thus every
  vector in $V$ is in the span of the linearly indepedent set $S$, and
  so $S$ is a basis.
\end{solution}

\begin{problem}
  Every vector space has a basis.
\end{problem}

\begin{solution}
  This can be proved by invoking the axiom of Choice, in the guise of
  Zorn's lemma.  (And in fact, this is equivalent to Choice.)
\end{solution}

\begin{problem}
  The kernel of the linear transformation $T : V \to W$ is a subspace of $V$.
\end{problem}

\begin{solution}
  The kernel is nonempty (because $\vec{0}$ is in $\ker T$), and a
  linear combination of vectors in the kernel is also in the kernel,
  because if $v_1, v_2 \in \ker T$, then
  \[
    T(\alpha_1 v_1 + \alpha_2 v_2) = \alpha_1 T(v_1) + \alpha_2 T(v_2) = 0.
  \]
\end{solution}

\begin{problem}
  The image of a basis is a basis.
\end{problem}

\begin{solution}
  This is not true, but it could be salvaged if, say, the
  transformation has trivial kernel.
\end{solution}

\begin{problem} % a couple possible salvages
  A linear transformation $T : V \to V$ is injective iff $\ker T = V$.
\end{problem}

\begin{solution}
  Again, this is wrong, but there are a couple possible salvages.  Injective could be changed to surjective (``a linear transformation $T : V \to V$ is \textit{surjective} iff $\ker T = V$'') or the condition on the kernel could be changed (a linear transformation $T : V \to V$ is injective iff $\ker T = \{ 0 \}$.).
\end{solution}

\begin{problem}
  A linear transformation $T : V \to V$ is either surjective or has nontrivial kernel.
\end{problem}

\begin{solution}
  These sorts of dichotomy results can be proved by invoking rank-nullity.  In this case, if $T$ is not surjective, then $\rank T < \dim V$, in which case $\nullity T > 0$, so $\ker T \neq \{ 0 \}$.
\end{solution}

\begin{problem}
  Suppose $T : V \to V$ is a linear transformation.  Then because of the rank-nullity theorem, $\ker T \cap \image T = \{0\}$.
\end{problem}

\begin{solution}
  This is designed to address a misconception that rank-nullity holds
  because dimensions of complementary subspaces sum to the total
  dimension.  The nicest salvage is to consider how large
  $\dim (\ker T \cap \image T)$ can be.
\end{solution}

\begin{problem}
  Suppose $f, g: V \to V$ are linear transformations with the property that $f \circ g = \id$.  Then $g \circ f = \id$.
\end{problem}

\begin{solution}
  This is not true if $V$ is infinite dimensional.  For instance, let
  $f$ be the shift operator, and define $g : \R^\infty \to \R^\infty$
  by $g(a_1,a_2,\ldots) = (0,a_1,a_2,\ldots)$.  Then $f \circ g = \id$
  but $(g \circ f)(a_1,a_2,\ldots) = (0,a_2,a_3,\ldots)$.

  But this is true with an added hypothesis: assume $V$ is finite
  dimensional.  Associativity shows
  $g = g \circ (f \circ g) = (g \circ f) \circ g$.  But
  $f \circ g = \id$ means $g$ is injective and so by the finite
  dimensionality of $V$, the operator $g$ is also surjective.
  Therefore, for every $v \in V$, there is a $w \in V$ so that
  $g(w) = v$, so $g(w) = (g \circ f)(g(w))$, and therefore
  $v = (g \circ f)(v)$.
\end{solution}

\begin{problem}
  Suppose $f : V \to V$ has the property for all $v \in V$ that $(f \circ f)(v) = 0$.  Then $\ker f \subset \image f$.
\end{problem}

\begin{solution}
  Salvage with $\image f \subset \ker f$.  Suppose $w \in \image f$.
  This means taht $w = f(v)$ for some $v \in V$.  But $f(f(v)) = 0$,
  so $f(v) \in \ker f$.  We have shown $\image f \subset \ker f$.

  What is stated (``$\ker f \subset \image f$'') need not be true.
  More emphatically, consider the zero map $f(v) = 0$.  In this case,
  $\ker f = V$ but $\image f = \{0\}$.
\end{solution}

\begin{problem}\label{simultaneous-diagonalizability-commute}Given
  linear operators $f, g : V \to V$ on a finite dimensional vector
  space $V$, suppose there is an ordered basis $\mathcal{B}$ of $V$ so
  that $[f]_{\mathcal{B}}$ and $[g]_{\mathcal{B}}$ are both diagonal
  matrices.  Then $f \circ g = g \circ f$.
\end{problem}

\begin{solution}
  Note that diagonal \textit{matrices} commute, and so
  \[
    [f]_{\mathcal{B}} \cdot [g]_{\mathcal{B}} = [g]_{\mathcal{B}} \cdot [f]_{\mathcal{B}}.
  \]
  Therefore
  \[
    [f \circ g]_{\mathcal{B}} = [g \circ f]_{\mathcal{B}},
  \]
  so $f \circ g = g \circ f$.
\end{solution}

\begin{problem}
  Given linear operators $f, g : V \to V$ on a finite dimensional
  vector space $V$ with the property $f \circ g = g \circ f$, there is
  an ordered basis $\mathcal{B}$ of $V$ so that $[f]_{\mathcal{B}}$
  and $[g]_{\mathcal{B}}$ are both diagonal matrices.
\end{problem}

\begin{solution}
  This is not true.  Just because two not-necessarily-diagonalizable
  operators on a finite dimension space commute does not mean that
  they are simultaneously diagonalizable though the converse is true
  and could be your salvage if you don't mind repeating
  \ref{simultaneous-diagonalizability-commute}: if two operators are
  simultaneously diagonalizable---meaning that there is a single
  ordered basis $\mathcal{B}$ so that $[f]_{\mathcal{B}}$ and
  $[g]_{\mathcal{B}}$ are both diagonal matrices---then
  $f \circ g = g \circ f$.

  In fact, if $f, g$ are diagonalizable, then commutativity is
  equivalent to being simultaneously diagonalizable.  So our search
  for a counterexample to this statement should proceed by seeking
  operators which are diagonalizable in the first place.  To emphasize this, pick $f = g$ but nilpotent, say represented by
  \[
    \begin{bmatrix}
      0 & 1 \\
      0 & 0
    \end{bmatrix}.
  \]
\end{solution}

\begin{problem}
  The vector space $V \cong \R^2$ has ordered basis $\mathcal{B} = (e_1,e_2)$ and the linear map $f : V \to V$ is represented by
  \[
    [f]_{\mathcal{B}} = \begin{bmatrix} a & b \\ 0 & c \end{bmatrix}.
  \]
  Then the minimal polynomial of $f$ is $(x-a)(x-c)$.
\end{problem}

\begin{solution}
  We first check, by explicit calculation, that $f$ satisfies $(x-a)(x-c)$.

  The only worry is that $f$ satisfies a lower degree monic
  polynomial, but in that case $f$ would be a multiple of the
  identity.  This happens if $a = c$ and $b = 0$, in which case the
  minimal polynomial of $f$ is $(x-a)$.  But aside from this case,
  we're done.
\end{solution}

\end{document}
