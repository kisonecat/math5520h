\documentclass{homework}
\course{Math 5520H}
\author{Jim Fowler}
\usepackage{amsmath}
\DeclareMathOperator{\Mat}{Mat}
\DeclareMathOperator{\End}{End}
\DeclareMathOperator{\Hom}{Hom}
\DeclareMathOperator{\id}{id}
\DeclareMathOperator{\image}{im}
\DeclareMathOperator{\Imag}{Imag}
\DeclareMathOperator{\rank}{rank}
\DeclareMathOperator{\nullity}{nullity}
\DeclareMathOperator{\trace}{tr}
\DeclareMathOperator{\Spec}{Spec}
\DeclareMathOperator{\Sym}{Sym}
\DeclareMathOperator{\pf}{pf}
\DeclareMathOperator{\Ortho}{O}

\newcommand{\C}{\mathbb{C}}

\DeclareMathOperator{\sla}{\mathfrak{sl}}
\newcommand{\norm}[1]{\left\lVert#1\right\rVert}
\newcommand{\transpose}{\intercal}


\begin{document}
\maketitle

\begin{inspiration}
I see the eigenvalue in thine eye, 
I hear the tender tensor in thy sigh. \\
Bernoulli would have been content to die,
Had he but known such $a^2 \cos 2 \phi$! \\
\byline{Stanislaw Lem, \textit{The Cyberiad}}
\end{inspiration}

\section{Terminology}

\begin{problem}
  What is an eigenvector?  An eigenvalue?
\end{problem}

\begin{problem}
  What is a \textit{generalized} eigenvector?
\end{problem}

\begin{problem}
  What is a complementary subspace?  (Relate this to \ref{direct-sum-of-spaces}.)
\end{problem}

\begin{problem}
  What is a nilpotent operator?
\end{problem}

\section{Numericals}

% some generalized eigenvectors computation?

\begin{problem}
  An operator $T : V \to V$ is \textbf{simple} if the only $T$-invariant subspaces are $\{0\}$ and $V$.  Is  $\begin{pmatrix} 0 & 1 \\ -1 & 0 \end{pmatrix}$ simple?  Does the field over which we work matter?
\end{problem}

\begin{problem}
  Find an invertible matrix $P$ so that
  \(
    P \begin{pmatrix}
      0 & 1 \\
      1 & 1 
    \end{pmatrix} P^{-1}
  \)
  is diagonal, and use this to produce a formula for $\begin{pmatrix}
      0 & 1 \\
      1 & 1 
    \end{pmatrix}^n$.  Compare your formula to \ref{fibonacci-sequence}.
\end{problem}

\begin{problem}
  A matrix $M = \left(\begin{array}{rrr}
\frac{1}{2} & 1 & 0 \\
0 & \frac{1}{2} & 1 \\
0 & 0 & \frac{1}{2}
\end{array}\right)$ is in Jordan canonical form.  Find a formula for $M^n$.
\end{problem}

\begin{problem}
  Compute the Jordan canonical form of
  $\begin{pmatrix} 1 & 1 \\ \epsilon & 1 \end{pmatrix}$ for a real number $\epsilon$.  What happens when $\epsilon = 0$?  When $\epsilon \neq 0$?
\end{problem}

\begin{problem}
  The linear transformation $T : \R^5 \to \R^5$ in the standard basis $\mathcal{B}$ is represented by
  \[
    [T]_{\mathcal{B}} = \left(\begin{array}{rrrrr}
            1 & 0 & 0 & 1 & 1 \\
            0 & 0 & 0 & 1 & 0 \\
            1 & 1 & 1 & 1 & 0 \\
            1 & 1 & 0 & 0 & 0 \\
            1 & 0 & 0 & 1 & 0
          \end{array}\right)
      \]
      and has minimal polynomial $x^{5} - 2x^{4} - 2x^{3} + 3x^{2} + x - 1$.  Would the minimal polynomial change if we had chosen a different basis?  Is $T$ invertible?  What is the minimal polynomial of $T^{-1}$?  Can you answer these questions without seeing a matrix for $T$?
\end{problem}

\begin{problem}
  A certain transformation $T : V \to V$ has minimal polynomial
  $x^5 + x - 1$.  Find a polynomial $p$ so that $T^{-1} = p(T)$.
\end{problem}

\section{Exploration}

\begin{problem}
  What are the eigenvalues of a nilpotent transformation?
\end{problem}

\begin{problem}
  Suppose $T : V \to V$ has the property that $T^2 = \id$.  Is it possible to decompose
  $V$ into a subspace $U$ and a complementary subspace $W$ so that
  $T |_U(v) = v$ and $T |_W(v) = -v$?
\end{problem}

\begin{problem}
  For a linear transformation $T : V \to V$, consider the sequence
  $a_k = \dim \ker T^k$.  Does it stabilize?  What if $V$ is finite
  dimensional?  Does this mean the sequence of vector
  spaces $V_k = \ker T^k$ eventually stabilizes?
\end{problem}

\begin{problem}
  Find an operator $T : V \to V$ with an eigenvalue $\lambda$ so that its algebraic multiplicity is very from its geometric multiplicity.  How far apart can you make these two quantities?
\end{problem}

\begin{problem}
  Develop the \textbf{polynomial functional calculus} in the following sense.  Suppose $p$ is a polynomial.  Then if $T : V \to V$ has eigenvalues $\lambda_1, \ldots, \lambda_n$, what are the eigenvalues of $p(T)$?  Begin by exploring the case for $T$ diagonalizable, and then apply your knowledge of Jordan canonical form.
\end{problem}

\begin{problem}
  Solve \ref{independent-exponentials} again by finding an operator for which the functions $f(x) = e^{\lambda x}$ and $g(x) = e^{\mu x}$ are distinct eigenvectors and invoking \ref{eigenvectors-independent}.  Can you show that $e^x$ and $xe^x$ are independent this way?
\end{problem}

\section{Prove or Disprove and Salvage if Possible}

\begin{problem}\label{eigenvectors-independent}
  Eigenvectors corresponding to distinct eigenvalues are linearly independent.
\end{problem}

\begin{problem}
  A transformation $T \in \Hom(V,V)$ has at most $\dim V$ eigenvalues.
\end{problem}

\begin{problem}
  Suppose $f, g \in \Hom(V,V)$ and $g$ is invertible.  Then $g \circ f \circ g^{-1}$ and $f$ have the same eigenvalues and eigenvectors.
\end{problem}

\begin{problem}
  Suppose $f,g \in \Hom(V,V)$ and $f \circ g = g \circ f$.  Then $f$ and $g$ have the same eigenvalues and eigenvectors.
\end{problem}

\begin{problem}\label{every-operator-satisfies-a-polynomial}
  For an operator $T : V \to V$, there is a nonzero polynomial $p$
  so that $p(T) = 0$.
\end{problem}

\begin{problem}
  The minimal polynomial (cf. \ref{definition-minimal-polynomial}) is well-defined.
\end{problem}

\begin{problem}
  The roots of the minimal polynomial of $T : V \to V$ are exactly the
  eigenvalues of the operator $T$.
\end{problem}

\begin{problem}
  For a finite dimensional vector space $V$, an operator $T : V \to V$
  is said to be \textbf{semi-simple} if for every $T$-invariant
  subspace, there is a complementary $T$-invariant subspace.  An
  operator is semi-simple if and only if it is diagonalizable.
\end{problem}


\end{document}

%%% Local Variables:
%%% mode: latex
%%% TeX-master: t
%%% End:
