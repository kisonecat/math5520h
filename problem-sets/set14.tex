\documentclass{homework}
\course{Math 5520H}
\author{Jim Fowler}
\usepackage{amsmath}
\DeclareMathOperator{\Mat}{Mat}
\DeclareMathOperator{\End}{End}
\DeclareMathOperator{\Hom}{Hom}
\DeclareMathOperator{\id}{id}
\DeclareMathOperator{\image}{im}
\DeclareMathOperator{\Imag}{Imag}
\DeclareMathOperator{\rank}{rank}
\DeclareMathOperator{\nullity}{nullity}
\DeclareMathOperator{\trace}{tr}
\DeclareMathOperator{\Spec}{Spec}
\DeclareMathOperator{\Sym}{Sym}
\DeclareMathOperator{\pf}{pf}
\DeclareMathOperator{\Ortho}{O}

\newcommand{\C}{\mathbb{C}}

\DeclareMathOperator{\sla}{\mathfrak{sl}}
\newcommand{\norm}[1]{\left\lVert#1\right\rVert}
\newcommand{\transpose}{\intercal}


\begin{document}
\maketitle

\begin{inspiration}
  ``It's full of contradictions\ldots So is independence!''  \byline{\textit{Hamilton: An American Musical} reminding us about proof by contradiction and linear independence.}
  %Cuando se acerca el fin, ya no quedan im\'agenes del recuerdo; s\'olo quedan palabras.
  %As the end approaches, there are no longer any images from memory---there are only words.
  %\byline{\textit{El inmortal}, Jorge Luis Borges}
\end{inspiration}

\section{Terminology}

\begin{problem}
  What is linear algebra?
\end{problem}

\begin{problem}
  What is a graph $G$?  A $d$-regular graph?  What is its adjacency matrix $A(G)$?
\end{problem}

\begin{problem}
  Let $G \subset \Hom(V,V)$.  What is $Z(G)$, the center of $G$?
\end{problem}

\section{Numericals}

\begin{problem}
  Linear algebra is useful for combinatorics.
  
  Suppose $\{ A_1, \ldots, A_k \}$ is a set of subsets of $\{1,2,\ldots,2n\}$, with the property that $\# A_i$ is odd, and if $i \neq j$ then $\# (A_i \cap A_j)$ is even.  How big can $k$ be?

  \textit{Hint:} Interpret this as a linear algebra problem over $\mathbb{F}_2$.
\end{problem}

\begin{problem}\label{trace-of-adjacency}What is the trace of $A(G)$?  What is the trace of $A(G)^2$?
\end{problem}

\begin{problem}
  Draw the graphs with adjacency matrices
  \[A(G) = \left(\begin{array}{rrrrr}
0 & 0 & 0 & 0 & 0 \\
0 & 0 & 1 & 0 & 1 \\
0 & 1 & 0 & 1 & 0 \\
0 & 0 & 1 & 0 & 1 \\
0 & 1 & 0 & 1 & 0
\end{array}\right)
\mbox{ and } A(H) = \left(\begin{array}{rrrrr}
0 & 1 & 1 & 1 & 1 \\
1 & 0 & 0 & 0 & 0 \\
1 & 0 & 0 & 0 & 0 \\
1 & 0 & 0 & 0 & 0 \\
1 & 0 & 0 & 0 & 0
                          \end{array}\right).
                      \]
                      What are the eigenvalues of these matrices?
\end{problem}

\section{Exploration}

\begin{problem}\label{hurwitz-problem}Recall $\Ortho(n)$ consists of $n$-by-$n$ orthogonal matrices.

  Note that for a unit vector $(a,b) \in \R^2$, the matrix $\displaystyle\begin{pmatrix} a & b \\ -b & a \end{pmatrix}$ is in $\Ortho(2)$.

  In this course, we used a ``doubling trick'' to complexify a real
  vector space.  Now use a ``doubling trick'' to produce a family of
  matrices in $\Ortho(4)$, the first row of which is a unit vector
  $(a,b,c,d) \in \R^4$, and for which each subsequent row is a signed
  permutation of that first row.  In other words, each row includes
  $\pm a$, $\pm b$, $\pm c$, $\pm d$ in some order.  (Geometrically,
  if you succeed you will have shown that the tangent bundle to $S^3$
  is trivial.)
\end{problem}

\begin{problem}
  Can you find a family of matrices as in \ref{hurwitz-problem} but for $\Ortho(3)$?
\end{problem}

\begin{problem}
  Another example of a group of matrices is
  \[
    H = \left\{ \begin{pmatrix} 1 & a & b \\ 0 & 1 & c \\ 0 & 0 & 1 \end{pmatrix} : a, b, c \in \R \right\}.
  \]
  This is the \textbf{Heisenberg group}.  As a vector space, $H \cong \R^3$, but there is additional structure to $H$ as a subset of $\Hom(\R^3,\R^3)$.

  If $A, B \in H$, show that $AB \in H$ and $A^{-1} \in H$.
  
  If $A, B \in H$, is it the case that $[A,B] \in H$?
\end{problem}

\begin{problem}
  What is the center of $H$?
\end{problem}

\begin{problem}
  Define $\exp : H \to H$ by $\exp(M) = e^M$.
  Is the map $\exp$ one-to-one?  Onto?
\end{problem}

\begin{problem}
  Throughout this course, we've seen an analogy between vector spaces
  (over a field $k$) and whole numbers.  The dictionary includes that
  \begin{center}\begin{tabular}{r@{ }l@{ }l}
    $1$  & corresponds to  & $k$, \\
    $n + m$ & corresponds to  & $V \oplus W$, \\
    $n \leq m$ & corresponds to  & $V \subseteq W$, \\
    $nm$ & corresponds to  & $V \otimes W$, and \\
    $n^m$ & corresponds to  & $V^{\otimes m}$.%, and \\
                  %$\binom{n}{m}$ & corresponds to  & $\bigwedge^m V$. \\
  \end{tabular}\end{center}
Can you push this analogy further?  What plays the role of
$\binom{n}{m}$?  What might play the role of $2^n$?
%(What would ``negative'' dimension mean? ``Fractional'' dimension?)
\end{problem}

\begin{problem}\label{tensor-of-graphs}For graphs $G$ and $H$, define a new graph ``$G \times H$'' with the property that
  \[
    A(G) \otimes A(H) = A(G \times H).
  \]
  Can you describe---in terms of graphs!---the vertices and edges of $G \times H$?
\end{problem}

\begin{problem}
  Is your definition in \ref{tensor-of-graphs} also the
  category-theoretic product?  That is, does a map from a graph
  $C$ to $A \times B$ correspond to a pair of maps $C \to A$ and
  $C \to B$?
\end{problem}

\vspace{-2ex}\begin{problem}
  Use \ref{tensor-of-graphs} and \ref{trace-of-tensor-product} to
  check \ref{trace-of-adjacency}.
\end{problem}

\begin{problem}Interpret $\trace A(G)^k$ in two ways.  One way would involve the eigenvalues
  of $A(G)$, and the other would involve paths on the graph.
\end{problem}

\section{Prove or Disprove and Salvage if Possible}

\begin{problem}
  The study of graphs is often through \textit{spectral} methods,
  i.e., we study a graph $G$ by studying the eigenvalues of $A(G)$.

  If $G$ is a $d$-regular graph, then the largest eigenvalue of $A(G)$ is $d$.
\end{problem}

\begin{problem}% this is false; salvage with at least one factor should be nonbipartite.
  The graph $G \times H$ is connected if and only if $G$ and $H$ are connected.
\end{problem}

\begin{problem}
  By regarding points in the plane as complex numbers, linear algebra may shed light on Euclidean geometry.
  
  If $z_1, \ldots, z_n \in \C$ satisfy $|z_i| \leq 1$, then
  \(
    \displaystyle\prod_{1 \leq i < j \leq n} |z_i - z_j|^2 \leq n^n
    \).
    
    \textit{Hint:} Relate this to Vandermonde matrices
    (\ref{vandermonde-determinant}) and apply Hadamard's inequality
    (\ref{hadamard-inequality}).  Could you rephrase this problem
    without complex numbers?  For what sorts of shapes is the maximum
    realized? % regular polygons
\end{problem}

\end{document}
