\documentclass{homework}
\course{Math 5520H}
\author{Jim Fowler}
\usepackage{tikz-cd}
\usepackage{amsmath}
\DeclareMathOperator{\Mat}{Mat}
\DeclareMathOperator{\End}{End}
\DeclareMathOperator{\Hom}{Hom}
\DeclareMathOperator{\id}{id}
\DeclareMathOperator{\image}{im}
\DeclareMathOperator{\Imag}{Imag}
\DeclareMathOperator{\rank}{rank}
\DeclareMathOperator{\nullity}{nullity}
\DeclareMathOperator{\trace}{tr}
\DeclareMathOperator{\Spec}{Spec}
\DeclareMathOperator{\Sym}{Sym}
\DeclareMathOperator{\pf}{pf}
\DeclareMathOperator{\Ortho}{O}

\newcommand{\C}{\mathbb{C}}

\DeclareMathOperator{\sla}{\mathfrak{sl}}
\newcommand{\norm}[1]{\left\lVert#1\right\rVert}
\newcommand{\transpose}{\intercal}


\usepackage[symbol]{footmisc}
\renewcommand{\thefootnote}{\fnsymbol{footnote}}

\begin{document}
\maketitle

\begin{inspiration}
  Motto for a research laboratory: What we work on today, others will first think of tomorrow.
  \byline{Alan J. Perlis} %in SIGPLAN Notices Vol. 17, No. 9, September 1982, pages 7 - 13.
\end{inspiration}

\section{Terminology}

\begin{problem}
  What is $V \otimes W$?  (Do \ref{tensor-versus-hom} or
  \ref{tensor-versus-bilinear-forms} provide a definition?)
\end{problem}

\begin{problem}
  What is the complexification $V^{\C}$?
\end{problem}

\begin{problem}
  What is $\bigwedge^n V$?
\end{problem}

%\begin{problem}
%  What is $\Sym^n V$?
%\end{problem}

\section{Numericals}

\begin{problem}
  What is $\dim (V \otimes U)$ in terms of $\dim V$ and $\dim U$?
\end{problem}

\begin{problem}
  Compute $\dim \bigwedge^k V$. % and $\dim \Sym^k V$.
\end{problem}

\begin{problem}
  Use the fact that $\pm \sqrt{n} \in \Spec \begin{pmatrix} 0 & n \\ 1 & 0 \end{pmatrix}$ and \ref{spectrum-of-tensor} to find a $4$-by-$4$ matrix having eigenvalue $\sqrt{a} + \sqrt{b}$.
\end{problem}

\begin{problem}\label{pfaffian-definition}The \textbf{Pfaffian} of a skew-symmetric $2n$-by-$2n$ matrix $M$ is written $\pf(M)$ and satisfies
  \[
    \frac{1}{n!} \omega^n = \pf(M) e_1 \wedge \cdots \wedge e_{2n},
  \]
  where $\omega = \displaystyle\sum_{1 \leq i < j \leq 2n} M_{i,j} e_i \wedge e_j$.

  In terms of the entries of $M$, find a formula for $\pf(M)$ when $M$
  is a $4$-by-$4$ matrix.
\end{problem}

\section{Exploration}

\begin{problem}\label{tensor-versus-hom}For finite dimensional vector spaces $V$ and $W$ defined over a
  field $k$, relate $\Hom(V,W)$ to $V^\star \otimes W$, meaning define
  a map
  \[
    F_{V,W} : \Hom(V,W) \to V^\star \otimes W
  \]
  and verify that it is isomorphism.
  
  Define $\trace : V^\star \otimes V \to k$ by the rule
  $\trace(f \otimes v) = f(v)$; in particular, is $\trace$ bilinear?

  For $M \in \Hom(V,V)$, does $\trace (F_{V,V} M)$ equal the trace of
  $M$?  
\end{problem}

\begin{problem}\label{trace-of-tensor-product}How do $\trace f$ and $\trace g$ relate to $\trace (f \otimes g)$?
\end{problem}

\begin{problem}\label{tensor-versus-bilinear-forms}For vector spaces $V, W, U$ defined over a field $k$, let $B(V,W)$
  consist of all bilinear maps $V \times W \to k$.  How does $B(V,W)$
  relate to $V \otimes W$?  What object is similarly related to
  $V \otimes W \otimes U$?
\end{problem}

\begin{problem}
  Let $U, V, W$ be finite dimensional vector spaces. Explain why \(\Hom(V^\star,W) \cong \Hom(W^\star,V)\) and explain why
  \[
    \Hom(\Hom(V^\star,W)^\star,U) \cong \Hom(V^\star,\Hom(W^\star,U)).
  \]
  Your explanation might connect these objects to various tensor products.
\end{problem}

\begin{problem}\label{symplectic-form-on-v-plus-v}A \textbf{symplectic form} is a non-degenerate skew-symmetric bilinear form\footnote{As usual, we work over a field of characteristic $\neq$ two.}.  Exhibit such a form on $V \oplus V^\star$.
\end{problem}

\begin{problem}
  We have seen \textbf{extension of scalars} such as complexification; there is the dual notion  of \textbf{restriction of scalars}.  In this case of $\R$ and $\C$, this ``change of rings'' begins with a complex vector space $W$ and produces a real vector space $W_\R$.
  
  Suppose $W$ is a complex inner product space; define $B : W_\R \otimes W_\R \to \R$ by the rule
  \[
    B(x,y) = \Imag \langle x, y \rangle,
  \]
  or in words, $B(x,y)$ is the imaginary part of the inner product of $x$ and $y$.   Is $B$ non-degenerate?  Skew-symmetric?  Bilinear?  How does $B$ relate to the form you found in \ref{symplectic-form-on-v-plus-v}?
\end{problem}

\begin{problem}
  In \ref{pfaffian-definition}, we computed $\pf(M)$ for a $4$-by-$4$
  skew-symmetric matrix.  In this case, compute $\pf(M)^2$ and compare
  it to $\det(M)$.  What conjectures can you make?
\end{problem}

\section{Prove or Disprove and Salvage if Possible}

\begin{problem}\label{spectrum-of-tensor}If $\lambda$ is an eigenvalue of $f : V \to V$ and $\mu$ is an
  eigenvalue of $g : W \to W$, then $\lambda \mu$ is an eigenvalue of
  $f \otimes g$ and $\lambda + \mu$ is an eigenvalue of $f \otimes \id_W + \id_V \otimes g$.
\end{problem}

\begin{problem}
  Every eigenvalue of $f \otimes g$ is of the form $\lambda \mu$ for
  $\lambda \in \Spec f$ and $\mu \in \Spec g$.
\end{problem}

\begin{problem}
  If $f : U \to V$ is injective, then $f \otimes f : U \otimes U \to V \otimes V$ is injective.
\end{problem}

\begin{problem}
  If $\dim V$ and $\dim W$ are finite and $f : V \to V$ and $g : W \to W$ are linear, then $\det (f \otimes g) = (\det f) \cdot (\det g)$.
\end{problem}

\begin{problem}
  Suppose $f : V \to W$ is a $\R$-linear map.  Then $f^{\C} : V^{\C} \to W^{\C}$ is a $\C$-linear map, and the diagram
  \[\begin{tikzcd}
      V \arrow{r}{f} \arrow{d}{j_V} & W \arrow{d}{j_W} \\
      V^{\C} \arrow{r}{f^{\C}}  & W^{\C}
    \end{tikzcd}\] commutes, meaning that
  $j_W \circ f = f^{\C} \circ j_V$.  (Compare this to
  \ref{natural-transformation-to-double-dual}.  What are the maps $j_V$ and $j_W$?  Do they depend on $f$?)
\end{problem}

\begin{problem}
  For vector spaces $V$ and $W$ over $\R$, there is an isomorphism \(\Hom_\R(V , W)^\C \cong \Hom_\C(V^\C,W^\C)\).
\end{problem}

\begin{problem}
  There is a symplectic form on a five-dimensional vector space.
%  If $W$ is a vector space with a symplectic form, then $\dim W$ is even.
\end{problem}


\end{document}
