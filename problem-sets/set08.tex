\documentclass{homework}
\course{Math 5520H}
\author{Jim Fowler}

\begin{document}
\maketitle

\begin{inspiration}
  The worst thing you can do is to completely solve a problem.
  \byline{Dan Kleitman}
\end{inspiration}

\section{Terminology}

\begin{problem}\label{defn-singular-point}What is a singular point?
\end{problem}

\section{Numericals}

\begin{problem}\label{euler-cauchy-distinct-roots}Find the general solution to \(x^{2}{\displaystyle\frac  {d^{2}y}{dx^{2}}}-4x{\displaystyle\frac  {dy}{dx}}+6y=0\).\\
  \textit{Hint:} use the Ansatz $y = x^m$.
\end{problem}

\begin{problem}\label{euler-cauchy-single-root}If we apply the Ansatz from
  \ref{euler-cauchy-distinct-roots} to find the general solution to
  \[ 
    x^{2}{\frac  {d^{2}y}{dx^{2}}}-5x{\frac  {dy}{dx}}+9y=0
  \]
  we find $y = x^3$ is a solution, but is there another solution?  \textit{Hint:} use  \textbf{reduction of order} to find another.
\end{problem}

\begin{problem}Recall \ref{introduction-airy-function} in which we met the \textbf{Airy's equation}
  \[
    f''(x) - x f(x) = 0.
  \]
  Use your ideas from \ref{recurrence-differential-analogy} to find a series solution $f(x) = \sum_{k=0}^\infty c_k x^k$ to Airy's equation.
\end{problem}


\begin{problem}Use \ref{more-ricatti} to find a couple solutions to
  \[
    x^3 y'(x) = x^4 \, y(x)^2 + 3 \, x^2 \, y(x) + 6.
  \]
\end{problem}

\section{Exploration}

\begin{problem}\label{second-order-euler-cauchy}Put together the knowledge you gained in \ref{euler-cauchy-distinct-roots} and \ref{euler-cauchy-single-root} to solve
  \[
    x^{2}{\frac {d^{2}y}{dx^{2}}}+ax{\frac {dy}{dx}}+by=0.
  \]
  This is a particular case of the \textbf{Euler--Cauchy equation},
  and is remarkable for being a second order linear differential
  equation that can be solved explicitly without resorting to special
  functions.
\end{problem}

\begin{problem}\label{riccati-equation}In the past, we've ``traded dimensions for derivatives'' (cf. \ref{introduction-phase-space}).  Other trade-offs are
  possible: here we study a first-order \textbf{Riccati equation} and
  note that although it is not linear, finding a solution can be
  reduced to solving a \textit{second-order} linear ordinary
  differential equation---precisely the sort of differential equation
  we are studying this week.

  Consider the nonlinear differential equation of the form
  \[
    y'(x)=A(x)\,y^{2}(x) + B(x)\,y(x) + C(x)
  \]
  where $A(x) \neq 0$ and $C(x) \neq 0$.  (If $C(x) = 0$, then this is
  a special case of \ref{bernoulli-equation}).

  Show that if $z(x) = A(x) \, y(x)$, then
  \[
    z'(x)=y^{2}(x) + D(x)\,z(x) + E(x)
  \]
  for $D(x) = B(x) + A'(x)/A(x)$ and $E(x) = A(x) \, C(x)$.
\end{problem}
  
\begin{problem}\label{more-ricatti}Continuing our above study of the Riccati equation,
  show that if $z(x) = -w'(x)/w(x)$, we have
  \[
    w''(x) - D(x) \, w'(x) + E(x) w(x) = 0.
  \]
\end{problem}

\begin{problem}
  A differential equation for a forced oscillator is
  $y''(x) + y(x) = f(x)$ where $f$ is the forcing function.  Because
  of resonance (\ref{resonance-example}), even if $f$ is bounded, $y$
  could be unbounded.  When results are unsatisfactory, change your
  hypothesis!  Suppose there is an $M$ so that for all $a, b \in \R$
  the integral $\int_a^b \left|f(x)\right| \, dx$ is less than $M$.
  Under this stronger hypothesis, must a solution to
  $y''(x) + y(x) = f(x)$ be bounded?
\end{problem}

\section{Prove or Disprove and Salvage if Possible}

\begin{problem}
  If $L(y) = y^{(n)} + \sum_{k=0}^{n-1} a_k y^{(k)}$ and $\varphi_1,\ldots,\varphi_n$ are solutions to $L(y) = 0$, then
  \[
    W'(\varphi_1,\ldots,\varphi_n)(x) = - a_{n-1}(x) W(\varphi_1,\ldots,\varphi_n)(x).
  \]
\end{problem}

\begin{problem}
  If $L(y) = y^{(n)} + \sum_{k=0}^{n-1} a_k y^{(k)}$ and 
  both $\{\varphi_1,\ldots,\varphi_n\}$ and $\{\psi_1,\ldots,\phi_n\}$ are bases of $\ker L$, then
  \[
    \frac{W(\varphi_1,\ldots,\varphi_n)(x)}{W(\psi_1,\ldots,\psi_n)(x)}
  \]
  is constant.
\end{problem}

\begin{problem}\label{legendre-polynomials}For integers $n \geq 0$, there is a \textit{polynomial} $P_n(x)$ with $P_{n}(1)=1$ and satisfying
  \textbf{Legendre's differential equation}
  \[
    \frac{d}{dx}\left(\left(1-x^{2}\right){\frac {d P_{n}(x)}{dx}}\right)+n(n+1)P_{n}(x)=0.
  \]
  These are the \textbf{Legendre polynomials.}
\end{problem}


\begin{problem}\label{sturm-separation}If $\varphi_1$ and $\varphi_2$ are linearly independent solutions to
  $y'' + c_1 y' + c_0 y = 0$, and $\varphi_1(a) = \varphi_1(b) = 0$
  and $\varphi_1$ does not vanish for $x \in (a,b)$, then $\varphi_2$
  has exactly one root in $(a,b)$.  (This is the \textbf{Sturm
    separation theorem} and you may be amused to look at a graph of
  the Airy function to appreciate the result.)
\end{problem}




\end{document}
