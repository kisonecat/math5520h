\documentclass{homework}
\course{Math 5520H}
\author{Jim Fowler}
\usepackage{amsmath}
\DeclareMathOperator{\Mat}{Mat}
\DeclareMathOperator{\End}{End}
\DeclareMathOperator{\Hom}{Hom}
\DeclareMathOperator{\id}{id}
\DeclareMathOperator{\image}{im}
\DeclareMathOperator{\Imag}{Imag}
\DeclareMathOperator{\rank}{rank}
\DeclareMathOperator{\nullity}{nullity}
\DeclareMathOperator{\trace}{tr}
\DeclareMathOperator{\Spec}{Spec}
\DeclareMathOperator{\Sym}{Sym}
\DeclareMathOperator{\pf}{pf}
\DeclareMathOperator{\Ortho}{O}

\newcommand{\C}{\mathbb{C}}

\DeclareMathOperator{\sla}{\mathfrak{sl}}
\newcommand{\norm}[1]{\left\lVert#1\right\rVert}
\newcommand{\transpose}{\intercal}


\begin{document}
\maketitle

\begin{inspiration}
A mathematician is a person who can find analogies between theorems; a better mathematician is one who can see analogies between proofs and the best mathematician can notice analogies between theories. One can imagine that the ultimate mathematician is one who can see analogies between analogies.
\byline{Stefan Banach}
\end{inspiration}

\section{Terminology}

\begin{problem}
  What is a metric space?
\end{problem}

\begin{problem}
  What is a contraction?
\end{problem}

\begin{problem}
  What does ``Lipschitz'' mean?
\end{problem}

\begin{problem}
  What is an autonomous differential equation?
\end{problem}

\section{Numericals}

\begin{problem}
  Sometimes a differential equation is \textbf{separable.}  For example, find a solution to
  \[
    \frac{dy}{dx} = \frac{y}{4x - x^2}.
  \]
\end{problem}

\begin{problem}\label{bernoulli-equation}An equation of the form
  \[
    \frac{dy}{dx} + p(x) \, y = q(x) \, y^n
  \]
  is a \textbf{Bernoulli differential equation} provided $n \neq 0, 1$.  Such a differential equation isn't linear, but the substitution $u = 1/y^{n-1}$ makes it linear.  As a special case of this technique, solve $y' = y - y^2$.
\end{problem}

\begin{problem}
  Compute
  \[
    \lim_{n \to \infty} \begin{pmatrix} 1/3 & 1/4 & 0 \\ 1/2 & 1/4 & 0 \\ 0 & 0 & 1 \end{pmatrix}^n \begin{pmatrix} x \\ y \\ 1 \end{pmatrix}
  \]
  using the Banach fixed-point theorem.
\end{problem}

\begin{problem}\label{picard-method-numerical}Consider iterates of the map
  \[
    S(f) = 1 + \int_{x_0}^x t \, f(t) \, dt.
  \]
  What is $S(0)$?  What is $S(S(0))$?  And $S(S(S(0)))$?  Relate your numerical calculations to a power series expansion.
\end{problem}

\begin{problem}
  What differential equation is being solved in \ref{picard-method-numerical}?
\end{problem}

\section{Exploration}

\begin{problem}
  Use \ref{introduction-phase-space} to relate solutions to $y'' + ay + b = 0$ as solutions to
  \[
    \mathbf{y}' = M\mathbf{y}
  \]
  for $M$ a two-by-two matrix related to $a$ and $b$.  Argue that $\mathbf{y} = e^{Mt}$ is a solution and compute $e^{Mt}$ in terms of the Jordan canonical form of $M$.
\end{problem}

\begin{problem}
  The special case in \ref{bernoulli-equation} is an example of an
  autonomous differential equation with $y' = f(y) = y - y^2$.  Note
  that $f(y) = 0$ when $y = 0$ or $y = 1$.  When $y$ is a solution to
  $y' = f(y)$, it seems that $\lim_{t \to \infty} f(t) = 1$.  Is this
  always the case?  How does the ``eventual'' behavior of a solution
  of $y' = f(y)$ relate to the zeroes of $f$?  (Such thinking will
  lead you to consider the \textbf{phase line}.)
\end{problem}

\begin{problem}\label{integrating-factor}In \ref{derivative-multiplication-commute}, we examined operators $\frac{d}{dt}$ and $M_x$.  Now consider these as operators on $C^\infty(\R)$ and consider $M_x$ more generally by defining $M_p : C^\infty(\R) \to C^\infty(\R)$ via the rule $M_p(f)(x) = p(x) f(x)$.

  Find a function $\mu$ so that \(
    M_\mu \circ (\displaystyle\frac{d}{dt} + M_p) = \displaystyle\frac{d}{dt} \circ M_\mu.
  \)
  Such a $\mu$ is called \textbf{an integrating factor.}
\end{problem}

\begin{problem}Use \ref{integrating-factor} to find a solution to
  \(
    \displaystyle\frac{dy}{dx} - \displaystyle\frac{y}{2} = 2 \sin (3t).
  \)
  How does the long-term behavior of your solution depend on the value $y(0)$?
\end{problem}

% fredholm integral operator, born approximation, etc.

\section{Prove or Disprove and Salvage if Possible}

\begin{problem}
  For the matrix $M = \begin{pmatrix} a & b \\ c & d \end{pmatrix}$, we have
    $\det (e^M) = e^{\trace M}$.
\end{problem}

\begin{problem}
  Suppose for square matrices $X$ and $Y$ we have $XY = YX$.  Then $e^X e^Y = e^{X+Y}$.
\end{problem}

\begin{problem}
  A differentiable function $f : \R \to \R$ is locally Lipschitz.
\end{problem}

\begin{solution}
  A \textit{continuously} differentiable function is locally Lipschitz.
\end{solution}

\begin{problem}
  If $f : X \to X$ is a contraction for a metric space $X$, then
  $f$ has a unique fixed point.
\end{problem}

\begin{solution}
  The completeness hypothesis is missing.
\end{solution}

\begin{problem}
  If $f : \R \to \R$ with $f(x) = x + \varphi(x)$ for $\varphi$ Lipschitz , then $f$ is surjective.
\end{problem}

\begin{solution}
  For a fixed $y$, define $S(x) = y - \varphi(x)$.  If $x$ is a fixed point of $S$, then $f(x) = y$.
\end{solution}

\begin{problem}
  A fixed point of the map
  \[
    \varphi \mapsto y_0 + \int_{x_0}^x f(t,\varphi(t)) \, dt
  \]
  is a solution to $y' = f(x,y)$.
\end{problem}

\begin{problem}
  The differential equation $\displaystyle\frac{dy}{dt} = y^{2/3}$ has a unique solution with $y(0) = 0$.
\end{problem}

\begin{problem}
  Because $x \mapsto x^2$ is defined for all $x \in \R$, there is a solution $y : \R \to \R$ to the differential equation $\displaystyle\frac{dy}{dt} = y^2$ with $y(0) > 0$.
\end{problem}

\end{document}
