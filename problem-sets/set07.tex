\documentclass{homework}
\course{Math 5520H}
\author{Jim Fowler}
\usepackage{amsmath}
\DeclareMathOperator{\Mat}{Mat}
\DeclareMathOperator{\End}{End}
\DeclareMathOperator{\Hom}{Hom}
\DeclareMathOperator{\id}{id}
\DeclareMathOperator{\image}{im}
\DeclareMathOperator{\Imag}{Imag}
\DeclareMathOperator{\rank}{rank}
\DeclareMathOperator{\nullity}{nullity}
\DeclareMathOperator{\trace}{tr}
\DeclareMathOperator{\Spec}{Spec}
\DeclareMathOperator{\Sym}{Sym}
\DeclareMathOperator{\pf}{pf}
\DeclareMathOperator{\Ortho}{O}

\newcommand{\C}{\mathbb{C}}

\DeclareMathOperator{\sla}{\mathfrak{sl}}
\newcommand{\norm}[1]{\left\lVert#1\right\rVert}
\newcommand{\transpose}{\intercal}

\usepackage{nicefrac}
\begin{document}
\maketitle

\begin{inspiration}
  We share a philosophy about linear algebra: we think basis-free, we
  write basis-free, but when the chips are down we close the office
  door and compute with matrices like fury.  \byline{Paul Halmos}
\end{inspiration}

\section{Terminology}

\begin{problem}
  Define determinant.  (How many different ``definitions'' can you give?)
\end{problem}

\begin{problem}
  What is a minor of a matrix?  How does it relate to a cofactor?
\end{problem}

\section{Numericals}

\begin{problem}\label{very-small-determinant}Compute, perhaps after looking at \ref{cauchy-determinant}, \(
    \det \displaystyle\begin{pmatrix}
\nicefrac{1}{1} & \nicefrac{1}{2} & \nicefrac{1}{3} & \nicefrac{1}{4} & \nicefrac{1}{5} \\
\nicefrac{1}{2} & \nicefrac{1}{3} & \nicefrac{1}{4} & \nicefrac{1}{5} & \nicefrac{1}{6} \\
\nicefrac{1}{3} & \nicefrac{1}{4} & \nicefrac{1}{5} & \nicefrac{1}{6} & \nicefrac{1}{7} \\
\nicefrac{1}{4} & \nicefrac{1}{5} & \nicefrac{1}{6} & \nicefrac{1}{7} & \nicefrac{1}{8} \\
\nicefrac{1}{5} & \nicefrac{1}{6} & \nicefrac{1}{7} & \nicefrac{1}{8} & \nicefrac{1}{9}
\end{pmatrix}.\) \\
\end{problem}

\begin{problem}
  Consider the vector space $V$ over $\mathbb{Q}$ with basis
  $ \mathcal{B} = \{1,\sqrt{2},\sqrt{3},\sqrt{6}\}$.  Write down a
  matrix representing the linear transformation
  $T(x) = (\sqrt{2} + \sqrt{3})x$ and find its characteristic
  polynomial $p(\lambda)$.  Compute $p(\sqrt{2} + \sqrt{3})$.
\end{problem}

\begin{problem}
  For a $3$-by-$3$ matrix $M$, the characteristic polynomial is given by
  \[
    \lambda^3 - \trace(M) \lambda^2 + f(M) \lambda - \det(M).
  \]
  Find a formula for $f(M)$ in terms of $\trace(M)$ and $\trace(M^2)$.
\end{problem}

\section{Exploration}

\begin{problem}\label{vandermonde-determinant}Study \textbf{Vandermonde matrices}, an instance of which appeared in \ref{vandermonde-introduction}.  Specifically, for $\alpha_1,\ldots,\alpha_n$, define
  \[
    V(\alpha_1,\ldots,\alpha_n) := \begin{pmatrix}
      1 & 1 & \cdots & 1 \\
      \alpha_1 & \alpha_2 & \cdots & \alpha_n \\
      {\alpha_1}^2 & {\alpha_2}^2 & \cdots & {\alpha_n}^2 \\
      \vdots & \vdots  &  & \vdots \\
      {\alpha_1}^{n-1} & {\alpha_2}^{n-1} & \cdots & {\alpha_n}^{n-1}
    \end{pmatrix}.
  \]
  When is $V(\alpha_1,\ldots,\alpha_n)$ nonsingular?  What is $\det V(\alpha_1,\ldots,\alpha_n)$?
\end{problem}

\begin{problem}\label{cauchy-determinant}A \textbf{Cauchy matrix} is a matrix $(a_{ij})$ with $a_{ij} = \displaystyle\frac{1}{x_i - y_j}$.  Compute the determinant of a square Cauchy matrix when the $x_i$ and $y_j$ are distinct.
\end{problem}

\begin{problem}
  An $n$-by-$n$ \textbf{circulant matrix} is a matrix of the form
  \[
    C(\alpha_1,\ldots,\alpha_n) := {\begin{pmatrix}\alpha_{0}&\alpha_{n-1}&\dots &\alpha_{2}&\alpha_{1}\\\alpha_{1}&\alpha_{0}&\alpha_{n-1}&&\alpha_{2}\\\vdots &\alpha_{1}&\alpha_{0}&\ddots &\vdots \\\alpha_{n-2}&&\ddots &\ddots &\alpha_{n-1}\\\alpha_{n-1}&\alpha_{n-2}&\dots &\alpha_{1}&\alpha_{0}\end{pmatrix}.}
  \]
  Find eigenvalues, eigenvectors, and the determinant of
  $C = C(\alpha_1,\ldots,\alpha_n)$.  \textit{Hint:} Consider the matrix $V = V(1,\omega,\ldots,\omega^{n-1})$ for a certain $\omega$, and compute $VCV^{-1}$ to achieve a  \textbf{discrete Fourier transform}.
\end{problem}

\begin{problem}
  Generalize circulant matrices.  To each element $g_i$ of a finite group $G = \{ g_1, \ldots, g_n \}$, associate an indeterminant $x_{g_i}$ and consider the matrix $(a_{ij})$ with $a_{ij} = x_{g_i {g_j}^{-1}}$.  Explore the \textbf{Frobenius determinant} $\det (a_{ij})$ in the special case that $G = \Z_2 \times \Z_2$, the Klein four-group.  How does  $\det (a_{ij})$ factor in this case?
\end{problem}

% permutations

%find formula for derivative of characteristic polynomial?

\section{Prove or Disprove and Salvage if Possible}

\begin{problem}% salvage to multiplicative
  For operators $A, B : V \to V$, the determinant is additive: $\det (A+B) = \det A + \det B$.
\end{problem}

\begin{problem}% salvage to handle multiplicity
  Suppose $T : V \to V$ has eigenvalues $\lambda_1,\ldots,\lambda_n$.
  Then $\det T = \prod_i \lambda_i$.
\end{problem}

\begin{problem} % salvage to 2^{n-1}
  Suppose $M = (m_{ij})$ is an $n$-by-$n$ matrix and $m_{ij} = \pm 1$.
  Then $\det M$ is a multiple of $2^{n}$.
\end{problem}

\begin{problem}
  Suppose $M$ is an $n$-by-$n$ matrix and consider its characteristic
  polynomial $p(\lambda) = \det(\lambda I - M)$.  Then $p(M) = 0$.  This is
  the \textbf{Cayley-Hamilton theorem.}
\end{problem}

\begin{problem}
  If $M$ is an $n$-by-$n$ matrix then there exists another $n$-by-$n$
  matrix $C$ so that $MC = CM = (\det M) I$.
\end{problem}

\begin{problem}
  If $A$ and $B$ are $n$-by-$n$ circulant matrices, then $AB$ is circulant and in fact $AB = BA$.
  \textit{Hint:} Can you show that $A = p(C)$ for a certain matrix $C$?
\end{problem}

\begin{problem}
  The rank of $M$ is the largest size of any non-zero minor of $M$.
\end{problem}

\begin{problem}
  Suppose the $2n$-by-$2n$ matrix $M$ splits into four $n$-by-$n$ blocks $\begin{pmatrix} A & B \\ C & D \end{pmatrix}$.

  Then $\det M = \det (AD - BC)$.
\end{problem}

\begin{problem}\label{sylvesters-determinant}If $A$ is an
  $m$-by-$n$ matrix and $B$ is an $n$-by-$m$ matrix, then
  $\det(\id_{m}+AB)=\det(\id_{n}+BA)$.  This is \textbf{Sylvester's
    determinant identity.}  \textit{Hint:} Compute
  \( \begin{pmatrix}\id_{m}&-A\\B&\id_{n}\end{pmatrix} \) in two
  different ways.
\end{problem}

\begin{problem}
  If $A$ and $B$ are $n$-by-$n$ matrices, then
  $\det(\lambda I - AB) = \det(\lambda I - BA)$, meaning $AB$ and $BA$
  have the same characteristic polynomial.
\end{problem}

\begin{problem}
  For a square matrix $A$, if $f(x) = \det(I + xA)$, then $f'(0) = \trace A$.
\end{problem}

\end{document}
