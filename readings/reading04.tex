\documentclass{homework}
\usepackage{amsmath}
\DeclareMathOperator{\Mat}{Mat}
\DeclareMathOperator{\End}{End}
\DeclareMathOperator{\Hom}{Hom}
\DeclareMathOperator{\id}{id}
\DeclareMathOperator{\image}{im}
\DeclareMathOperator{\Imag}{Imag}
\DeclareMathOperator{\rank}{rank}
\DeclareMathOperator{\nullity}{nullity}
\DeclareMathOperator{\trace}{tr}
\DeclareMathOperator{\Spec}{Spec}
\DeclareMathOperator{\Sym}{Sym}
\DeclareMathOperator{\pf}{pf}
\DeclareMathOperator{\Ortho}{O}

\newcommand{\C}{\mathbb{C}}

\DeclareMathOperator{\sla}{\mathfrak{sl}}
\newcommand{\norm}[1]{\left\lVert#1\right\rVert}
\newcommand{\transpose}{\intercal}

\usepackage{hyperref}
\author{Jim Fowler}
\course{Math 5520H}
\title{Assigned Readings for Week 4}
\date{Monday, September 10, 2018}
\begin{document}
\maketitle

For this week, we begin the study of \textit{differential equations}.
Recall that at the beginning of this course---back in Week 1---we
studied systems of linear equations.  Both ``differential equations''
and ``linear equations'' include the word ``equations'' but the
connection between these two subjects certainly runs much deeper than
this.  Be prepared to make analogies and connections!

To get an overview of the subject, start with
\href{/courses/43735/files/folder/textbooks}{Linear Algebra and
  Differential Equations using MATLAB} and read
\begin{itemize}
\item \textsection 4.1 A Single Differential Equation
\item \textsection 4.2 Graphing Solutions to Differential Equations
\item \textsection 4.3 Phase Space Pictures and Equilibria
\end{itemize}
These readings will provide some geometric insight.

Our computations for this week will focus on linear equations with
constant coefficients.  Skim through the beginning of \textit{An
  Introduction to Ordinary Differential Equations} by Coddington, and
then focus on
\begin{itemize}
\item \textsection 2.2 The second order homogeneous equation
\item \textsection 2.3 Initial value problems for second order equations
\item \textsection 2.4 Linear dependence and independence
\item \textsection 2.5 A formula for the Wronskian
\item \textsection 2.6 The non-homogeneous equation of order two
\item \textsection 2.7 The homogeneous equation of order $n$
\item \textsection 2.8 Initial value problems for $n$-th order equations
\end{itemize}

Your problem set for this week will make an analogy between shift
operators and differential equations, and I hope you will spend time
reflecting on the power of linear algebra in providing a framework
broad enough to encompass so much ``linear'' phenomena.  Next week,
motivated by our study of differential equations, we meet eigenvectors
for the first time.

I should emphasize that the ``interweaving'' of topics we are
attempting in this course is both challenging and
important. Gian-Carlo Rota once wrote: ``The sophomore course in
differential equations will never be reformed.  It will die of natural
death, and it will be replaced by several shorter courses that will
deal with realistic aspects of differential equations.''  We will revisit differential equations over the remainder of the semester, and each week should be regarded as a short course covering a ``realistic'' aspect of differential equations.

\end{document}
