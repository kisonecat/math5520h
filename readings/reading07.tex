\documentclass[11pt]{homework}
\usepackage{amsmath}
\DeclareMathOperator{\Mat}{Mat}
\DeclareMathOperator{\End}{End}
\DeclareMathOperator{\Hom}{Hom}
\DeclareMathOperator{\id}{id}
\DeclareMathOperator{\image}{im}
\DeclareMathOperator{\Imag}{Imag}
\DeclareMathOperator{\rank}{rank}
\DeclareMathOperator{\nullity}{nullity}
\DeclareMathOperator{\trace}{tr}
\DeclareMathOperator{\Spec}{Spec}
\DeclareMathOperator{\Sym}{Sym}
\DeclareMathOperator{\pf}{pf}
\DeclareMathOperator{\Ortho}{O}

\newcommand{\C}{\mathbb{C}}

\DeclareMathOperator{\sla}{\mathfrak{sl}}
\newcommand{\norm}[1]{\left\lVert#1\right\rVert}
\newcommand{\transpose}{\intercal}

\usepackage{hyperref}
\author{Jim Fowler}
\course{Math 5520H}
\title{Assigned Readings for Week 7}
\date{Monday, October 1, 2018}
\setlength{\itemsep}{0in}
\begin{document}
\maketitle

As our journey comes to its middle, we return our attention to
\textit{differential equations}, and you may enjoy searching the
internet for Gian-Carlo Rota's recommendations for ``reforms'' to the
second-year differential equations course, and in particular his
feelings about the sorts of theorems we are focusing on this week.

And just what are we focusing on?  This week, we return again to
initial value problems, but this time we look at the
\textbf{Picard-Lindel\"of theorem}, the statement of which ensures
that \textbf{Picard's method} of successive approximations actually
converges to a solution.  This content is in Chapter~5 of Coddington's
differential equations text, so look in \textit{An Introduction to
  Ordinary Differential Equations} and focus on
\begin{itemize}
\item \textsection 5.4 The method of successive approximations
\item \textsection 5.5 The Lipschitz condition
\item \textsection 5.6 Convergence of the successive approximations
\item \textsection 5.7 Non-local existence of solutions
\item \textsection 5.8 Approxmations to, and uniqueness of, solutions
\end{itemize}
These iterative methods are built on a foundation of the
\textbf{Banach fixed-point theorem}, and formulating that will lead us
to consider \textbf{metric spaces,} an object you may be familiar with
from your work in analysis.

As usual, there is some ``computation'' task for us each week, and these computations lie mainly with
\begin{itemize}
  \item \textsection 5.2 Equations with variables separated
  \item \textsection 5.3 Exact equations
\end{itemize}
which will provide you with some additional techniques for practically
solving certain differential equations.

To connect the differential equations story back to linear algebra,
look at \textit{\href{/courses/43735/files/folder/textbooks}{Linear
    Algebra and Differential Equations using MATLAB}} and note how
Jordan canonical form relates to solving differential equations.
Matrix exponentials play a particular role.  Specific sections to look
at include
\begin{itemize}
\item \textsection 6.1 The Initial Value Problem
\item \textsection 6.2 Closed Form Solutions by the Direct Method
\item \textsection 6.3 Solutions Using Matrix Exponentials
\item \textsection 6.4 Linear Jordan Normal Form Planar Systems
\item \textsection 6.5 Similar Matrices and Jordan Normal Form
\item \textsection 6.6 The Cayley Hamilton Theorem
\item \textsection 6.7 Second Order Equations  
\end{itemize}

\end{document}
