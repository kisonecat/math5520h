\documentclass{homework}
\usepackage{amsmath}
\DeclareMathOperator{\Mat}{Mat}
\DeclareMathOperator{\End}{End}
\DeclareMathOperator{\Hom}{Hom}
\DeclareMathOperator{\id}{id}
\DeclareMathOperator{\image}{im}
\DeclareMathOperator{\Imag}{Imag}
\DeclareMathOperator{\rank}{rank}
\DeclareMathOperator{\nullity}{nullity}
\DeclareMathOperator{\trace}{tr}
\DeclareMathOperator{\Spec}{Spec}
\DeclareMathOperator{\Sym}{Sym}
\DeclareMathOperator{\pf}{pf}
\DeclareMathOperator{\Ortho}{O}

\newcommand{\C}{\mathbb{C}}

\DeclareMathOperator{\sla}{\mathfrak{sl}}
\newcommand{\norm}[1]{\left\lVert#1\right\rVert}
\newcommand{\transpose}{\intercal}

\usepackage{hyperref}
\author{Jim Fowler}
\course{Math 5520H}
\title{Assigned Readings for Week 5}
\date{Monday, September 17, 2018}
\begin{document}
\maketitle

After having introduced differential equations last week, we noticed
that a single second-order differential equation is related to a
coupled pair of first-order equations, and this inspires us to think
more deeply about $2 \times 2$ matrices, to consider eigenvectors and
eigenvalues in the $2 \times 2$ case, and finally to consider the case
of a linear operator $f : V \to V$.

You will also be studying for \textbf{Midterm 1} this week and yet I
want to make sure that you have time to digest the ``eigenstuff.''
Therefore next week we will be considering ``generalized
eigenvectors'' and we will have a chance to revisit this week's
material in depth.

As is the usual plan of attack, we rely on differential equations as a
venue to motivate our study of linear algebra.  To make connections
between differential equations and the eigenvectors and eigenvalues of
$2 \times 2$ matrices, take a look at
\textit{\href{/courses/43735/files/folder/textbooks}{Linear Algebra
    and Differential Equations using MATLAB}}, namely
\begin{itemize}
\item \textsection 4.5 Uncoupled Linear Systems of Two Equations
\item \textsection 4.6 Coupled Linear Systems
\item \textsection 4.7 The Initial Value Problem and Eigenvectors
\item \textsection 4.8 Eigenvalues of $2\times 2$ Matrices
\item \textsection 4.9 Initial Value Problems Revisited  
\end{itemize}

To move into the general case of ``eigenstuff'' for an operator
$f : V \to V$, study \textit{Linear Algebra Done Right} by Sheldon
Axler, specifically in Chapter~5.  This material is also in
\textit{Linear Algebra: An Introductory Approach}, specifically in
\begin{itemize}
\item \textsection 22 Basic concepts
\item \textsection 23 Invariant subspaces
\end{itemize}
but I prefer Axler's introduction to eigenvectors and eigenvalues.
Axler downplays the role of determinants, and I'd prefer that your
desire to find invariant subspaces establish an intellectual need for
determinants, rather than your simply being handed ``$\det$'' and
running with it.  Reflect on the difference between the minimal
polynomial and the characteristic polynomial.

\textit{Warning:} It is possible you have concerns about your
knowledge of polynomials over the complex numbers $\mathbb{C}$.
Chapter~4 of \textit{Linear Algebra Done Right} and \textsection 20
and \textsection 21 of \textit{Linear Algebra: An Introductory
  Approach} both provide a review of polynomials.

\end{document}
