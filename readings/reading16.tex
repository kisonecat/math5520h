\documentclass{homework}
\usepackage{amsmath}
\DeclareMathOperator{\Mat}{Mat}
\DeclareMathOperator{\End}{End}
\DeclareMathOperator{\Hom}{Hom}
\DeclareMathOperator{\id}{id}
\DeclareMathOperator{\image}{im}
\DeclareMathOperator{\Imag}{Imag}
\DeclareMathOperator{\rank}{rank}
\DeclareMathOperator{\nullity}{nullity}
\DeclareMathOperator{\trace}{tr}
\DeclareMathOperator{\Spec}{Spec}
\DeclareMathOperator{\Sym}{Sym}
\DeclareMathOperator{\pf}{pf}
\DeclareMathOperator{\Ortho}{O}

\newcommand{\C}{\mathbb{C}}

\DeclareMathOperator{\sla}{\mathfrak{sl}}
\newcommand{\norm}[1]{\left\lVert#1\right\rVert}
\newcommand{\transpose}{\intercal}

\usepackage{hyperref}
\author{Jim Fowler}
\course{Math 5520H}
\title{Assigned Readings for Week 16}
\date{Monday, December 3, 2018}
\begin{document}
\maketitle

%Read ``the calf-path'' by Sam Foss.  Your job is not to repeat
%history, but to be (in the words of Yuri on Ice) a history maker.

We began our course months ago with an overview, noting that our
subject could be seen from many different perspectives.  We end our
journey with an underview.  Having done the hard work of reading
through the various ``Assigned Readings,'' you have perhaps learned
something of other people's perspectives---but most importantly you
have discovered your \textit{own} perspective on our subject.

Having your own perspective on differential equations and linear
algebra prepares you to dig into other books over winter break.  There
are many possible directions.  Never stop learning!
\begin{description}
\item[Differential geometry.] Perhaps the most natural push-out of
  differential equations and linear algebra is towards differential
  geometry.  Ted Shifrin's notes \textit{Differential Geometry: A
    First Course in Curves and Surfaces} are available freely online.
\item[Quaternions.] Before there were vectors, there were quaternions.
  We had a brief glimpse of quaternions when we considered a
  ``doubling'' trick for complexification of real vector spaces, and
  wondered what would happened if we doubled the complex numbers.  A
  fun---and short---book on such topics is John Conway's \textit{On
    Quaternions and Octonions}.
\item[Fourier series.] When solving differential equations, think of
  how frequently $\sin(n\theta)$ and $\cos(n\theta)$ appeared.
  Decomposing functions into sums of sines and cosines is the purview
  of Fourier analysis, and a truly enjoyable (seriously, novel-like!)
  introduction is T. W. K\"orner's \textit{Fourier analysis}.  The
  book is quite long, but it is build around short essays, and
  contains quite a bit of interesting history about, e.g., Lord
  Kelvin.
\item[Linear algebra and groups.] Study a group by thinking of its
  elements as acting on a vector space, by ``representing'' a group as
  a group of matrices.  Making this connection between symmetry and
  linear algebra is an instance of ``representation theory'' which you
  can read about in Serre's \textit{Linear Representations of Finite
    Groups}.  The book is quite short.
\item[Metrics and groups.] Did you enjoy our use of metric spaces, such
  as when we used the Banach fixed point theorem?  How about combining
  metric spaces and groups!  A readable introduction to this topic is
  Clara L\"oh's \textit{Geometric group theory, an introduction}.
\item[Category theory.] If you liked thinking about direct sums via
  universal properties, you might enjoy more category theory.  Emily
  Riehl's \textit{Category Theory in Context} has the benefit of being
  well-written and modern.
\end{description}
Whatever you end up doing with your knowledge of mathematics, be sure
that you use it not for evil but for good.  ``For the Quest is
achieved, and now all is over.  I am glad you are here with me.  Here
at the end of all things, Sam.'' % -- The Return of the King

\end{document}
