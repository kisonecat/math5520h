\documentclass{homework}
\usepackage{amsmath}
\DeclareMathOperator{\Mat}{Mat}
\DeclareMathOperator{\End}{End}
\DeclareMathOperator{\Hom}{Hom}
\DeclareMathOperator{\id}{id}
\DeclareMathOperator{\image}{im}
\DeclareMathOperator{\Imag}{Imag}
\DeclareMathOperator{\rank}{rank}
\DeclareMathOperator{\nullity}{nullity}
\DeclareMathOperator{\trace}{tr}
\DeclareMathOperator{\Spec}{Spec}
\DeclareMathOperator{\Sym}{Sym}
\DeclareMathOperator{\pf}{pf}
\DeclareMathOperator{\Ortho}{O}

\newcommand{\C}{\mathbb{C}}

\DeclareMathOperator{\sla}{\mathfrak{sl}}
\newcommand{\norm}[1]{\left\lVert#1\right\rVert}
\newcommand{\transpose}{\intercal}

\usepackage{hyperref}
\author{Jim Fowler}
\course{Math 5520H}
\title{Assigned Readings for Week 2}
\date{Monday, August 26, 2019}
\begin{document}
\maketitle

For our second week together, our goal is to move away from systems of
linear equations, from examples involving $\R^n$, and towards vector
spaces in general.

Last week we viewed matrices as an organizing device for systems of
linear equations.  ``Rank'' was a significant notion on last week's
problem set.  We perhaps noticed that, when interpreting solutions of
linear equations geometrically along with our intuitive idea of
dimension, we found a connection between dimension and rank.

It is time to make those observations more precise: this week we
provide definitions of dimension (among other things) and turn our
attention to the multiplication of matrices---revealing that matrices
are more than merely a book-keeping device.  (If you are impatient,
next week we will view matrices as encoding \textit{linear
  transformations}.)  Findnig the inverse of a nonsingular matrix is
another computational goal for this week.

To develop your understanding of vector spaces in general, look at
\textit{Linear Algebra: An Introductory Approach} and study
\begin{itemize}
\item \textsection 3 Vector spaces
\item \textsection 4 Subspaces and linear dependence
\item \textsection 5 The concepts of basis and dimension
\item \textsection 7 Some general theorems about finitely generated vector spaces
\end{itemize}
At the speed with which this course moves, it is \textbf{extremely
  important} that you are keeping up with the reading.

Another (shorter!  terser!) optional reference for this material is
Chapters~1 and~2 of \textit{Linear Algebra Done Right} by Sheldon
Axler.

If you want to see a book that highlights matrices (as opposed to
linear transformations), you may enjoy the book \textit{Matrix
  Analysis} by Roger Horn and Charles Johnson.

\end{document}
