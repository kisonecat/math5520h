\documentclass{homework}
\usepackage{amsmath}
\DeclareMathOperator{\Mat}{Mat}
\DeclareMathOperator{\End}{End}
\DeclareMathOperator{\Hom}{Hom}
\DeclareMathOperator{\id}{id}
\DeclareMathOperator{\image}{im}
\DeclareMathOperator{\Imag}{Imag}
\DeclareMathOperator{\rank}{rank}
\DeclareMathOperator{\nullity}{nullity}
\DeclareMathOperator{\trace}{tr}
\DeclareMathOperator{\Spec}{Spec}
\DeclareMathOperator{\Sym}{Sym}
\DeclareMathOperator{\pf}{pf}
\DeclareMathOperator{\Ortho}{O}

\newcommand{\C}{\mathbb{C}}

\DeclareMathOperator{\sla}{\mathfrak{sl}}
\newcommand{\norm}[1]{\left\lVert#1\right\rVert}
\newcommand{\transpose}{\intercal}

\usepackage{hyperref}
\author{Jim Fowler}
\course{Math 5520H}
\title{Assigned Readings for Week 2}
\date{Monday, August 27, 2018}
\begin{document}
\maketitle

For our second week together, our goal is to move away from examples
involving $\R^n$ and towards vector spaces in general.

More specifically, last week we viewed matrices primarily as an
organizing device for systems of linear equations and we interpreted
solutions of linear equations geometrically through our intuitive idea
of dimension, which seemed to align in various ways with ``rank.''
This week, we provide definitions of dimension (among other things)
and turn our attention to the multiplication of matrices---revealing
that matrices are more than merely a book-keeping device.  (If you are
impatient, next week we will view matrices as encoding \textit{linear
  transformations}.)

To develop your understanding of vector spaces in general, look at
\textit{Linear Algebra: An Introductory Approach} and study
\begin{itemize}
\item \textsection 3 Vector spaces
\item \textsection 4 Subspaces and linear dependence
\item \textsection 5 The concepts of basis and dimension
\item \textsection 7 Some general theorems about finitely generated vector spaces
\end{itemize}
Another (and terser) recommended reference for this material is
Chapters~1 and~2 of \textit{Linear Algebra Done Right} by Sheldon
Axler.  At the speed with which this course moves, it is extremely
important that you are reading.

To gain a practical appreciation of matrices and matrix
multiplication, look at Chapter 3 of
\textit{\href{/courses/43735/files/folder/textbooks}{Linear Algebra
    and Differential Equations using MATLAB}}, namely
\begin{itemize}
\item \textsection 3.1 Matrix Multiplication of Vectors
\item \textsection 3.2 Matrix Mappings
\item \textsection 3.3 Linearity
\item \textsection 3.4 The Principle of Superposition
\item \textsection 3.5 Composition and Multiplication of Matrices
\item \textsection 3.6 Properties of Matrix Multiplication
\item \textsection 3.7 Solving Linear Systems and Inverses
\end{itemize}
If you want to see a book that highlights matrices (as opposed to
linear transformations), you may enjoy \textit{Matrix Analysis} by
Roger Horn and Charles Johnson.


\end{document}
