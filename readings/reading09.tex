\documentclass{homework}
\usepackage{amsmath}
\DeclareMathOperator{\Mat}{Mat}
\DeclareMathOperator{\End}{End}
\DeclareMathOperator{\Hom}{Hom}
\DeclareMathOperator{\id}{id}
\DeclareMathOperator{\image}{im}
\DeclareMathOperator{\Imag}{Imag}
\DeclareMathOperator{\rank}{rank}
\DeclareMathOperator{\nullity}{nullity}
\DeclareMathOperator{\trace}{tr}
\DeclareMathOperator{\Spec}{Spec}
\DeclareMathOperator{\Sym}{Sym}
\DeclareMathOperator{\pf}{pf}
\DeclareMathOperator{\Ortho}{O}

\newcommand{\C}{\mathbb{C}}

\DeclareMathOperator{\sla}{\mathfrak{sl}}
\newcommand{\norm}[1]{\left\lVert#1\right\rVert}
\newcommand{\transpose}{\intercal}

\usepackage{hyperref}
\author{Jim Fowler}
\course{Math 5520H}
\title{Assigned Readings for Week 9}
\date{Monday, October 15, 2018}
\begin{document}
\maketitle

After our short week in which we met the many faces of the
determinant, we can now return to differential equations and to the
study of the Wronskian.

Draw some inspiration from gazing upon ``phase portraits'' and reading
\textit{\href{/courses/43735/files/folder/textbooks}{Linear Algebra
    and Differential Equations using MATLAB}}, focusing on
\begin{itemize}
\item \textsection 7.1 Sinks, Saddles, and Sources
\item \textsection 7.2 Phase Portraits of Sinks
\item \textsection 7.3 Phase Portraits of Nonhyperbolic Systems
\end{itemize}
Indeed, Gian-Carlo Rota writes ``In an elementary course in
differential equations, students should learn a few basic concepts
that they will remember for the rest of their lives, such as the
universal occurrence of the exponential function, stability, the
relationship between trajectories and integrals of systems, phase
plane analysis, the manipulation of the Laplace transform, perhaps
even the fascinating relationship between partial fraction
decompositions and convolutions via Laplace transforms.  Who cares
whether the students become skilled at working out tricky problems?''

Then, despite Gian-Carlo Rota's advice, look to the case of linear
equations with variable coefficients, which is in Chapter~3 of
\textit{An Introduction to Ordinary Differential Equations} by
Coddington.  For this, focus on
\begin{itemize}
\item \textsection 3.2 Initial value problems for the homogeneous equation
\item \textsection 3.3 Solutions of the homogeneous equation % 1, 2
\item \textsection 3.4 The Wronskian and linear independence % 1
\item \textsection 3.5 Reduction of the order of a homogeneous equation % 1(ac) 4
\item \textsection 3.6 The non-homogeneous equation % 1
\item \textsection 3.7 Homogeneous equations with analytic coefficients % 2, 5
\item \textsection 3.8 The Legendre equation % 9
\end{itemize}
We are all very happy to see the Wronskian again, and remarkably many
of its previous wonderful properties will be reprised in this context.

Finally, you may be amused to look back at some very old texts and
consider how much is unchanged.  Some to consider are Abraham Cohen's
\textit{An elementary treatise on differential equations} from 1906,
or George Boole's \textit{A treatise on differential equations} from
1859.  What would you say to a student in 2118 being told to read
Coddington's text?

Finally, for those of you who are interested, I have a research
problem involving Legendre polynomials.

\end{document}
