\documentclass{homework}
\usepackage{amsmath}
\DeclareMathOperator{\Mat}{Mat}
\DeclareMathOperator{\End}{End}
\DeclareMathOperator{\Hom}{Hom}
\DeclareMathOperator{\id}{id}
\DeclareMathOperator{\image}{im}
\DeclareMathOperator{\Imag}{Imag}
\DeclareMathOperator{\rank}{rank}
\DeclareMathOperator{\nullity}{nullity}
\DeclareMathOperator{\trace}{tr}
\DeclareMathOperator{\Spec}{Spec}
\DeclareMathOperator{\Sym}{Sym}
\DeclareMathOperator{\pf}{pf}
\DeclareMathOperator{\Ortho}{O}

\newcommand{\C}{\mathbb{C}}

\DeclareMathOperator{\sla}{\mathfrak{sl}}
\newcommand{\norm}[1]{\left\lVert#1\right\rVert}
\newcommand{\transpose}{\intercal}

\usepackage{hyperref}
\author{Jim Fowler}
\course{Math 5520H}
\title{Assigned Readings for Week 15}
\date{Monday, November 26, 2018}
\begin{document}
\maketitle

My goal in crafting this course is to provide some scaffolding on
which you will build your own understanding and learning.  Providing
such scaffolding is especially important when there is a challenge
shared by many students, like the challenge of getting a ``feeling''
for the \textbf{tensor product} $\otimes$.  So this week, we build on
our understanding of quotients and duals and bilinear forms to look
carefully at the tensor product.

To prepare, turn to \textit{Linear Algebra: An Introductory Approach}
and review
\begin{itemize}
\item \textsection 26 Quotient spaces and dual vector spaces
\item \textsection 27 Bilinear forms and duality
\end{itemize}
which provide a foundation for our learning this week.  Then continue
with
\begin{itemize}
\item \textsection 28 Direct sums and tensor products
\end{itemize}
which introduces the main subject.  If you would prefer to think about the tensor product more generally for $R$-modules, look at Keith Conrad's notes
\begin{itemize}
\item \href{http://www.math.uconn.edu/~kconrad/blurbs/linmultialg/tensorprod.pdf}{\textit{Tensor Products}} and
\item \href{http://www.math.uconn.edu/~kconrad/blurbs/linmultialg/tensorprod2.pdf}{\textit{Tensor Products II}}.
\end{itemize}

One application of the tensor product is \textbf{complexification},
which is itself a special case of \textbf{extension of scalars}
permitting a vector space over a field $K$ to be viewed as a vector
space over a larger field $L > K$.  A good source to read about
complexification is Axler's \textit{Linear Algebra Done Right},
specifically
\begin{itemize}
\item \textsection 9.A Complexification
\end{itemize}
and again Keith Conrad's short note \href{http://www.math.uconn.edu/~kconrad/blurbs/linmultialg/complexification.pdf}{\textit{Complexifications}} is a great source.

The tensor product is perhaps the first object a student studies whose
clearest definition is a universal property rather than an explicit
construction.  Think back, for instance, to $V \oplus W$.  If we want
a construction, we might think of $V \oplus W$ as pairs of vectors
(the first from $V$, the second from $W$).  But we could also view
$V \oplus W$ as the result of a certain universal property.  This
week, we'll see a ``construction'' of the tensor product as well, but
I suspect that you'll find the construction less satisfying.
What's really going on is that the tensor product is a gadget from
\textbf{multilinear algebra} which lets us regard a bilinear map
$f : V \times W \to U$ as a linear map---but the price we pay is a
more complicated domain.

And paying the price of such abstraction, there had better be some
sort of reward.  We'll finally put the \textbf{wedge product} written
$\wedge$ on a firm foundation (giving a nice perspective on the
\textbf{determinant}) which we'll extend by learning about the
\textbf{Pfaffian}.  Our past learning about inner products which will
deepened by considering \textbf{symplectic forms}.  My hope is that a
little early exposure to such ideas will make your future travels
through mathematics a bit easier.

\end{document}
