\documentclass{homework}
\usepackage{amsmath}
\DeclareMathOperator{\Mat}{Mat}
\DeclareMathOperator{\End}{End}
\DeclareMathOperator{\Hom}{Hom}
\DeclareMathOperator{\id}{id}
\DeclareMathOperator{\image}{im}
\DeclareMathOperator{\Imag}{Imag}
\DeclareMathOperator{\rank}{rank}
\DeclareMathOperator{\nullity}{nullity}
\DeclareMathOperator{\trace}{tr}
\DeclareMathOperator{\Spec}{Spec}
\DeclareMathOperator{\Sym}{Sym}
\DeclareMathOperator{\pf}{pf}
\DeclareMathOperator{\Ortho}{O}

\newcommand{\C}{\mathbb{C}}

\DeclareMathOperator{\sla}{\mathfrak{sl}}
\newcommand{\norm}[1]{\left\lVert#1\right\rVert}
\newcommand{\transpose}{\intercal}

\usepackage{hyperref}
\author{Jim Fowler}
\course{Math 5520H}
\title{Assigned Readings for Week 10}
\date{Monday, October 22, 2018}
\begin{document}
\maketitle

Almost two months ago we saw our first glimpse of \textbf{inner
  products.}  It's been a while, so open \textit{Linear Algebra: An
  Introductory Approach} and review \textsection 15 Inner products.  You may also want to consult Axler's \textit{Linear Algebra Done Right} and read
\begin{itemize}
\item \textsection 6.A Inner Products and Norms
\item \textsection 6.B Orthonormal Bases
\item \textsection 6.C Orthonormal Complements and Minimization Problems
\item \textsection 7.A Self-Adjoint and Normal Operators
\end{itemize}
Note that last topic: we will study self-adjoint operators which
suggests by lecture theory that we will study \textbf{adjoint
  operators}\footnote{Last week, we met Cramer's rule, which involved a
certain matrix of cofactors which nowadays is called the
\textbf{adjugate matrix} but \textit{used} to be called the adjoint.
These things have (or rather had\ldots) the same name, but they're not
related, and similar names for different things are often a point of
confusion.}.  Inner products will lead us to consider \textbf{norms} on vector
spaces, and in particular to prove that all norms on finite
dimensional spaces are equivalent.  We will see the \textbf{Schur decomposition} and the \textbf{QR decomposition} as well as unitary and Hermitian matrices.

Thinking about inner products also leads us to \textbf{bilinear forms}
and to \textbf{linear functionals,} which  reminds us about
\textbf{dual spaces}.  For more on dual spaces, open \textit{Linear
  Algebra: An Introductory Approach} and read
\begin{itemize}
\item \textsection 26 Quotient spaces and dual vector spaces
\item \textsection 27 Bilinear forms and duality
\end{itemize}
You will also find these topics in \textit{Linear Algebra Done Right}
in \textsection 3.E Products and Quotients of Vector Spaces and
\textsection 3.F Duality.  Pay special attention to the finite dimensional \textbf{Riesz representation theorem}.

In addition to this ``conceptual'' content, there is, as usual, a ``computational'' goal for the week.  For this week, be prepared to apply the \textbf{Gram-Schmidt procedure.}  To see some additional applications, the textbook
\textit{\href{/courses/43735/files/folder/textbooks}{Linear Algebra
    and Differential Equations using MATLAB}} includes a discussion of
least squares.  Take a look at
\begin{itemize}
\item \textsection 10.1 Orthonormal Bases
\item \textsection 10.2 Least Squares Approximations
\item \textsection 10.3 Least Squares Fitting of Data
\item \textsection 10.4 Symmetric and Orthogonal Matrices
\end{itemize}


\end{document}
