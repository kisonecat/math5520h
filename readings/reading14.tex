\documentclass{homework}
\usepackage{amsmath}
\DeclareMathOperator{\Mat}{Mat}
\DeclareMathOperator{\End}{End}
\DeclareMathOperator{\Hom}{Hom}
\DeclareMathOperator{\id}{id}
\DeclareMathOperator{\image}{im}
\DeclareMathOperator{\Imag}{Imag}
\DeclareMathOperator{\rank}{rank}
\DeclareMathOperator{\nullity}{nullity}
\DeclareMathOperator{\trace}{tr}
\DeclareMathOperator{\Spec}{Spec}
\DeclareMathOperator{\Sym}{Sym}
\DeclareMathOperator{\pf}{pf}
\DeclareMathOperator{\Ortho}{O}

\newcommand{\C}{\mathbb{C}}

\DeclareMathOperator{\sla}{\mathfrak{sl}}
\newcommand{\norm}[1]{\left\lVert#1\right\rVert}
\newcommand{\transpose}{\intercal}

\usepackage{hyperref}
\author{Jim Fowler}
\course{Math 5520H}
\title{Assigned Readings for Week 14}
\date{Monday, November 19, 2018}
\begin{document}
\maketitle

A course in mathematics isn't just a sequence of theorems; it is a
narrative that weaves definitions and theorems together into a story.
Our story this semester has woven together linear algebra and
differential equations, with examples like the constant coefficient
case playing a key role in motivating the study of both.

We finally take up \textbf{Sturm–Liouville
  theory}.  We've seen hints and shadows of this during our semester
together, so this week provides a chance to connect and consolidate
our past learning.  Specifically, we'll investigate differential equations of the form
\begin{equation*}\label{sturm-liouville-form}\tag{$*$}
{ {\frac {d }{d x}}\left( p(x){\frac { {d} y}{ {d} x}}\right)+q(x)y=-\lambda w(x)y,} \end{equation*}
subject to certain \textbf{boundary conditions}.

Why is this interesting?  For starters, many of the equations we have studied previously can be placed in the form~\eqref{sturm-liouville-form}.  For instance, the \textbf{Bessel equation}
\[  x^{2}y''+xy'+\left(x^{2}-n^{2}\right)y=0
\]
could have been written in the form~\eqref{sturm-liouville-form} as
\[
  {\displaystyle \left(xy'\right)'+\left(x-{\frac {n ^{2}}{x}}\right)y=0,}
\]
and the \textbf{Legendre equation}
\[
{\displaystyle \left(1-x^{2}\right)y''-2xy'+n (n +1)y=0}
\]
can be written as
\[
  {\displaystyle \left(\left(1-x^{2}\right)y'\right)'+n (n +1)y=0}.
\]
But interest in Sturm–Liouville theory also lies in its connections to other mathematics.  Last week, for instance, we met spectral theorems, and there are likewise spectral theorems for Sturm-Liouville equations.  When $w \equiv 1$, the form~\eqref{sturm-liouville-form} becomes
\[
  \left( p(x) y'(x) \right)'+q(x)y=-\lambda y(x),
\]
or in other words, $-\lambda$ is an eigenvalue for the operator given by $Ly = \left( p \, y' \right)'+q \, y$.  Is $L$ self-adjoint?  Are eigenfunctions with different eigenvalues orthogonal?  Are the eigenvalues real?  

This is a topic which isn't covered as well in our usual texts.  One
textbook that I can recommend is a newer (2008) Springer book by
Mohammed Al-Gwaiz entitled \textit{Sturm-Liouville Theory and its
  Applications}, particularly the two dozen pages of \textsection 2.4
The Sturm–Liouville Problem.  The rest of the book addresses the
singular case.  The ``other'' Coddington textbook (\textit{Linear
  Ordinary Differential Equations} by Coddington and Carlson) also
discusses Sturm-Liouville theory.

%https://www.rose-hulman.edu/~bryan/lottamath/complete2.pdf

\end{document}
