\documentclass{homework}
\usepackage{amsmath}
\DeclareMathOperator{\Mat}{Mat}
\DeclareMathOperator{\End}{End}
\DeclareMathOperator{\Hom}{Hom}
\DeclareMathOperator{\id}{id}
\DeclareMathOperator{\image}{im}
\DeclareMathOperator{\Imag}{Imag}
\DeclareMathOperator{\rank}{rank}
\DeclareMathOperator{\nullity}{nullity}
\DeclareMathOperator{\trace}{tr}
\DeclareMathOperator{\Spec}{Spec}
\DeclareMathOperator{\Sym}{Sym}
\DeclareMathOperator{\pf}{pf}
\DeclareMathOperator{\Ortho}{O}

\newcommand{\C}{\mathbb{C}}

\DeclareMathOperator{\sla}{\mathfrak{sl}}
\newcommand{\norm}[1]{\left\lVert#1\right\rVert}
\newcommand{\transpose}{\intercal}

\usepackage{hyperref}
\author{Jim Fowler}
\course{Math 5520H}
\title{Assigned Readings for Week 3}
\date{Tuesday, September 4, 2018}
\begin{document}
\maketitle

For our third week together, we view matrices not as arrays of
numbers, but rather as encoding linear transformations---the
structure-preserving functions between vector spaces.  In so doing, we
must also refine our understanding of a ``basis'' and view a basis not
merely as a device for computing dimension, but as providing
coordinates for the vectors in $V$.  Through the choice of bases,
computations with linear transformations are as easy as matrix
multiplication.  And as we study maps between vector spaces, we also
introduce new structure on the vector spaces themselves, namely an
inner product.  For motivation, recall your previous experience with
the ``dot product.''

Look at \textit{Linear Algebra: An Introductory Approach} and focus on
\begin{itemize}
\item \textsection 11 Linear transformations
\item \textsection 12 Addition and multiplication of matrices
\item \textsection 13 Linear transformations and matrices
\end{itemize}
A shorter overview this material is Chapter~3 of \textit{Linear
  Algebra Done Right} by Sheldon Axler.  Last week, I mentioned
another book I quite like, \textit{Finite Dimensional Vector Spaces}
by Halmos; if you want to look at that book, explore Part II,
particularly \textsection 32--38.

As mentioned earlier, in addition to getting our feet wet with linear
transformations, we'll also introduce ``inner products'' this week but we won't make much progress on this topic.
From \textit{Linear Algebra: An Introductory Approach}, take a peek at
\textsection 15 Inner products.  Alternatively, look at Chapter~6 of
\textit{Linear Algebra Done Right} by Sheldon Axler.  This discussion
will only begin late this week, and it will bleed over into next week.

As usual, each week, in addition to introducing some ``conceptual''
content, I also want you to be comfortable with certain
``computational'' content.  So if you've never seen determinants
before, it will be helpful for you to be able to perform such
computations.  From
\href{/courses/43735/files/folder/textbooks}{Linear Algebra and
  Differential Equations using MATLAB}, read \textsection 3.8
Determinants of $2\times 2$ Matrices.  You may be amused to note that
\textit{Linear Algebra Done Right} pushes determinants off until the
very end of text!  Determinants certainly bring out some strong
emotions.

\end{document}
