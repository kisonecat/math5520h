\documentclass{homework}
\usepackage{amsmath}
\DeclareMathOperator{\Mat}{Mat}
\DeclareMathOperator{\End}{End}
\DeclareMathOperator{\Hom}{Hom}
\DeclareMathOperator{\id}{id}
\DeclareMathOperator{\image}{im}
\DeclareMathOperator{\Imag}{Imag}
\DeclareMathOperator{\rank}{rank}
\DeclareMathOperator{\nullity}{nullity}
\DeclareMathOperator{\trace}{tr}
\DeclareMathOperator{\Spec}{Spec}
\DeclareMathOperator{\Sym}{Sym}
\DeclareMathOperator{\pf}{pf}
\DeclareMathOperator{\Ortho}{O}

\newcommand{\C}{\mathbb{C}}

\DeclareMathOperator{\sla}{\mathfrak{sl}}
\newcommand{\norm}[1]{\left\lVert#1\right\rVert}
\newcommand{\transpose}{\intercal}

\usepackage{hyperref}
\author{Jim Fowler}
\course{Math 5520H}
\title{Assigned Readings for Week 8}
\date{Monday, October 8, 2018}
\begin{document}
\maketitle

\begin{inspiration}
  Old MacDonald had a form, $e_i \wedge e_i = 0$. \byline{Ravi Vakil}
\end{inspiration}

Last week, we considered a general existence and uniqueness theorem
for \textit{first order} differential equations; next week, we'll
study the case of \textit{higher order} linear differential equations
with variable coefficients.  But if we expect next week to prove that
we have $n$ linearly independent solutions, we'll need a tool for
proving the linear independence of many solutions\ldots like the
Wronskian?  So during this shortened week we'll develop
\textbf{determinants}, giving us access to the Wronskian next week.

One complaint that you may have about this course is the somewhat
strange order in which we are interweaving topics.  In response, let
me include some text from one of my heroes, Gian-Carlo Rota, who
writes:
\begin{quote}
  The facts of mathematics are verified and presented by the axiomatic
  method. One must guard, however, against confusing the presentation
  of mathematics with the content of mathematics. An axiomatic
  presentation of a mathematical fact differs from the fact that is
  being presented as medicine differs from food. It is true that this
  particular medicine is necessary to keep mathematicians at a safe
  distance from the self-delusions of the mind. Nonetheless,
  understanding mathematics means being able to forget the medicine
  and enjoy the food.
\end{quote}
Although the readings will expose you to the usual axiomatic treatment
for determinants, I hope that in our all-too-short time together you
also gain an appreciation for the power and ``fun'' of determinants.

Here history should be our guide: yes, we feel happy when we have
defined determinants not of a matrix but of a linear operator, and
perhaps we are happiest when thinking about wedge products\ldots but
determinants appear in the work of Seki Takakazu in 1683, long before
anyone on earth was talking about ``linear operators'' let alone wedge
products.  How is this possible?

%The Nine Chapters on the Mathematical Art (simplified Chinese: 九章算术; traditional Chinese: 九章算術; pinyin: Jiǔzhāng Suànshù)

%解伏題之法 / Kaifukudai no hō
Determinants and matrices, it seems, are more fundamental than we may
have guessed; perhaps our desire to be basis-free sometimes prejudices
us.  To dig into determinants, read \textit{Linear Algebra: An
  Introductory Approach} and look at
\begin{itemize}
\item \textsection 16 Definition of determinants
\item \textsection 17 Existence and uniqueness of determinants
\item \textsection 18 The multiplication theorem for determinants
\item \textsection 19 Further properties of determinants
\end{itemize}
You may want to look at Axler's \textit{Linear Algebra Done Right} and
reflect on Axler's choice to downplay the role of determinants.

\end{document}
