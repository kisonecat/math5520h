\documentclass{homework}
\usepackage{amsmath}
\DeclareMathOperator{\Mat}{Mat}
\DeclareMathOperator{\End}{End}
\DeclareMathOperator{\Hom}{Hom}
\DeclareMathOperator{\id}{id}
\DeclareMathOperator{\image}{im}
\DeclareMathOperator{\Imag}{Imag}
\DeclareMathOperator{\rank}{rank}
\DeclareMathOperator{\nullity}{nullity}
\DeclareMathOperator{\trace}{tr}
\DeclareMathOperator{\Spec}{Spec}
\DeclareMathOperator{\Sym}{Sym}
\DeclareMathOperator{\pf}{pf}
\DeclareMathOperator{\Ortho}{O}

\newcommand{\C}{\mathbb{C}}

\DeclareMathOperator{\sla}{\mathfrak{sl}}
\newcommand{\norm}[1]{\left\lVert#1\right\rVert}
\newcommand{\transpose}{\intercal}

\usepackage{hyperref}
\author{Jim Fowler}
\course{Math 5520H}
\title{Assigned Readings for Week 12}
\date{Monday, November 5, 2018}
\begin{document}
\maketitle

At the end of this week, we will have \textbf{Midterm~2}, and since
you will also be studying for this challenging exam this week, I want
to make sure that you have time to reflect on your past learning and
that you have time to make connections between the linear algebra and
differential equations content.

The exam is built  from eight questions, consisting of one
definition, three numerical computations, two exploration problems,
and two ``prove or disprove, and salvage if possible'' prompts.
\textbf{Be sure to provide a response to every question asked} because
the only way to earn points is to provide responses.
This exam is cumulative, but of course overwhelmingly focuses on the
content covered since Midterm~1.  \textbf{In particular, any ``new''
  material from this week will not appear on Midterm~2}, so the
upcoming exam will focus on material from the ``eigenstuff'' in Week~5
all the way to the techniques for solving differential equations with
regular singular points.

Our time in lecture will focus on digging into some linear algebra
which we have seen shadows of, but not yet discussed in depth in
class.  One of these topics are various constructions for new vector
spaces; we've seen the direct sum $V \oplus W$, and we've seen how
$\mathrm{Hom}(V,W)$ is a vector space, but another construction is
that of \textbf{quotient spaces}, which you can review in Axler's
\textit{Linear Algebra Done Right} in \textsection 3.E Products and
Quotients of Vector Spaces, or in Curtis' \textit{Linear Algebra: An
  Introductory Approach} in \textsection 26 Quotient spaces and dual
vector spaces.

\textbf{Duality} is a topic we've seen, but a deeper study will
provide opportunities to reinforce your knowledge of linear algebra.
From \textit{Linear Algebra Done Right}, study \textsection 3.F
Duality; from \textit{Linear Algebra: An Introductory Approach}, study
the aforementioned \textsection 26 and also \textsection 27 Bilinear
forms and duality.  For a ``computational'' review task this week,
recall how to start with some reasonable basis (say, a basis of
monomials for polynomials of bounded degree) and find a dual basis.

Through duality, we will be led to review our short-lived experience
with \textbf{indices} which will lead us to brief foray into tensor
products $V \otimes W$ and into \textbf{biilinear maps}, making
connections to inner product spaces and perhaps even to \textbf{wedge
  products}, which we first mentioned when thinking about a basis-free
description of the \textbf{determinant.}

At the end of the week, armed with all this, we will study linear
functionals on spaces of polynomials, i.e., \textbf{the umbral
  calculus}, so-named for its power in proving ``shadowy'' identities
involving polynomials, e.g., behold the syntactic similarity between
\[
 \frac {d}{dx} x^{n}=n\,x^{n-1} \mbox{ and } {\frac {d}{dx}}B_{n}(x)=n \, B_{n-1}(x),
\]
where $B_n$ denotes the $n$th Bernoulli polynomial.

\end{document}
