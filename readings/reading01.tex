\documentclass{homework}
\usepackage{amsmath}
\DeclareMathOperator{\Mat}{Mat}
\DeclareMathOperator{\End}{End}
\DeclareMathOperator{\Hom}{Hom}
\DeclareMathOperator{\id}{id}
\DeclareMathOperator{\image}{im}
\DeclareMathOperator{\Imag}{Imag}
\DeclareMathOperator{\rank}{rank}
\DeclareMathOperator{\nullity}{nullity}
\DeclareMathOperator{\trace}{tr}
\DeclareMathOperator{\Spec}{Spec}
\DeclareMathOperator{\Sym}{Sym}
\DeclareMathOperator{\pf}{pf}
\DeclareMathOperator{\Ortho}{O}

\newcommand{\C}{\mathbb{C}}

\DeclareMathOperator{\sla}{\mathfrak{sl}}
\newcommand{\norm}[1]{\left\lVert#1\right\rVert}
\newcommand{\transpose}{\intercal}

\usepackage{hyperref}
\author{Jim Fowler}
\course{Math 5520H}
\title{Assigned Readings for Week 1}
\date{Tuesday, August 21, 2018}
\begin{document}
\maketitle

Linear algebra is cursed with being presentable from many
perspectives.  Some textbooks emphasize algebra, while others
emphasize geometric notions.  In this course---what with its being an
honors course---I expect you to not only learn linear algebra, but
also to learn to appreciate our subject from a variety of
perspectives.  I also expect you to be independent learners, meaning
that you must be willing to dig into a variety of sources despite the
differing notation and goals of each source.  Each week, an ``Assigned
Readings'' list like this page will provide such sources.

In light of these many sources, your attention to the Problem Sets
becomes critical.  The Problem Sets provide the common ground uniting
the disparate perspectives we find in sources outside of our
classroom.  And perhaps our classroom is the best source: rather than
looking things up in books, I'd prefer you work out the truth for
yourselves.  Ultimately, your goal is not to transfer knowledge from
textbooks to your mind, but rather for our class to develop the key
ideas together.

For this first week, our specific goal is to understand systems of
equations, matrices as an organizing device for such systems, and how
Gaussian elimination might help us solve systems of equations.  This
should feel leisurely, but there are many opportunities for deeper
conversations.

From \textit{\href{/courses/43735/files/folder/textbooks}{Linear Algebra and
  Differential Equations using MATLAB}}, begin with
\begin{itemize}
\item \textsection 2.1 Systems of Linear Equations and Matrices
\item \textsection 2.2 The Geometry of Low-Dimensional Solutions
\item \textsection 2.3 Gaussian Elimination
\item \textsection 2.4 Reduction to Echelon Form
\item \textsection 2.5 Linear Equations with Special Coefficients
\item \textsection 2.6 Uniqueness of Reduced Echelon Form
\end{itemize}

From \textit{Linear Algebra: An Introductory Approach}, take a look at
\begin{itemize}
\item \textsection 1 Some problems which lead to linear algebra
\item \textsection 2 Number systems and mathematical induction
\item \textsection 6 Row equivalence of matrices
\item \textsection 8 Systems of linear equations
\end{itemize}

\end{document}
