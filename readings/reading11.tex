\documentclass{homework}
\usepackage{amsmath}
\DeclareMathOperator{\Mat}{Mat}
\DeclareMathOperator{\End}{End}
\DeclareMathOperator{\Hom}{Hom}
\DeclareMathOperator{\id}{id}
\DeclareMathOperator{\image}{im}
\DeclareMathOperator{\Imag}{Imag}
\DeclareMathOperator{\rank}{rank}
\DeclareMathOperator{\nullity}{nullity}
\DeclareMathOperator{\trace}{tr}
\DeclareMathOperator{\Spec}{Spec}
\DeclareMathOperator{\Sym}{Sym}
\DeclareMathOperator{\pf}{pf}
\DeclareMathOperator{\Ortho}{O}

\newcommand{\C}{\mathbb{C}}

\DeclareMathOperator{\sla}{\mathfrak{sl}}
\newcommand{\norm}[1]{\left\lVert#1\right\rVert}
\newcommand{\transpose}{\intercal}

\usepackage{hyperref}
\author{Jim Fowler}
\course{Math 5520H}
\title{Assigned Readings for Week 11}
\date{Monday, October 29, 2018}
\begin{document}
\maketitle

Having studied inner products last week, we're in a stronger position
to think about orthogonality.  And indeed, this week we'll see more
examples of \textbf{orthogonal polynomials} like the \textbf{Laguerre
  polynomials}, which are solutions to
\[
  x y'' + (1-x) \, y' + n \, y = 0.
\]
This is a linear differential equation with nonconstant coefficients,
so we should not be scared as we have already studied that case in
great detail.  And yet, the leading term is not constant---and at some
points like $x = 0$, the leading term vanishes!  That more worrying
behavior of \textbf{singular points} forms our main topic for this
week.

We will see some examples where this is bad news like
\[
  x^2 y'' - y' - y = 0
\]
but we will quickly focus on the case of \textbf{regular singular
  points} where the \textbf{Frobenius method} and \textbf{Fuchs'
  theorem} apply.

The easiest case is one that we've already seen, namely the
\textbf{Euler-Cauchy equation} which we'll generalize this week.
Recall that the trick was to find solutions of the form $x^r$ for $r$
not necessarily an integer.  Mixing that idea with the idea of series
solutions is basically the \textbf{Frobenius method} which we'll try
in a few examples.

For details, look to Chapter~4 of Coddington's \textit{An Introduction
  to Ordinary Differential Equations} which consists of
\begin{itemize}
\item \textsection 4.1 Introduction
\item \textsection 4.2 The Euler equation
\item \textsection 4.3 Second order equations with regular singular points---an example
\item \textsection 4.4 Second order equations with regular singular points---the general case
\item \textsection 4.5 A convergence proof
\item \textsection 4.6 The exceptional cases
\item \textsection 4.7 The Bessel equation
\item \textsection 4.8 The Bessel equation (continued)
\item \textsection 4.9 Regular singular points at infinity
\end{itemize}
Note that we'll meet some ``famous'' functions, like the Bessel
functions.  Don't be too distracted by the individual examples,
because I especially want you to make connections and see the
analogies between the many different ``famous'' examples, ultimately
getting a preview of \textbf{Sturm-Liouville theory.}

\end{document}
