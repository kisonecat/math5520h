\documentclass{homework}
\usepackage{amsmath}
\DeclareMathOperator{\Mat}{Mat}
\DeclareMathOperator{\End}{End}
\DeclareMathOperator{\Hom}{Hom}
\DeclareMathOperator{\id}{id}
\DeclareMathOperator{\image}{im}
\DeclareMathOperator{\Imag}{Imag}
\DeclareMathOperator{\rank}{rank}
\DeclareMathOperator{\nullity}{nullity}
\DeclareMathOperator{\trace}{tr}
\DeclareMathOperator{\Spec}{Spec}
\DeclareMathOperator{\Sym}{Sym}
\DeclareMathOperator{\pf}{pf}
\DeclareMathOperator{\Ortho}{O}

\newcommand{\C}{\mathbb{C}}

\DeclareMathOperator{\sla}{\mathfrak{sl}}
\newcommand{\norm}[1]{\left\lVert#1\right\rVert}
\newcommand{\transpose}{\intercal}

\usepackage{hyperref}
\author{Jim Fowler}
\course{Math 5520H}
\title{Assigned Readings for Week 13}
\date{Tuesday, November 13, 2018}
\begin{document}
\maketitle

Just as we followed \textbf{Midterm \#1} with more linear algebra, so
too will \textbf{Midterm \#2} be followed with more linear algebra.
Indeed, on the midterm you were faced with the following challenge:
\begin{quote}
  Suppose $V$ is a finite dimensional inner product space over the
  complex numbers, and $T : V \to V$ is self-adjoint.  Show that there
  is an orthonormal basis of $V$ consisting of eigenvectors for which
  the corresponding eigenvalues are real.
\end{quote}
This is an example of a \textbf{spectral theorem}, i.e., in this case
conditions which imply that an operator can be \textbf{diagonalized}.
The reason that such a result belonged on the midterm is precisely
because such results are central to our story relating differential
equations and linear algebra: recall the frequency with which we
showed that functions were orthogonal with respect to a certain inner
product, and sometimes the ``moral'' for such orthogonality was that a
self-adjoint differential operator was lurking in the background.

From Axler's \textit{Linear Algebra Done Right}, read (or review)
\begin{itemize}
\item \textsection 7.A Self-Adjoint and Normal Operators
\item \textsection 7.B The Spectral Theorem
\item \textsection 7.C Positive Operators and Isometries
\item \textsection 7.D Polar Decomposition and Singular Value Decomposition
\end{itemize}
You can also look at \textit{Linear Algebra: An
  Introductory Approach}, specifically \textsection 32 Unitary transformations and the spectral theorem.

In both these sources, you will discover that a result like that from
the midterm applies not only to operators satisfying $M^\star = M$,
but more generally to operators with the property that
$M^\star M = M M^\star$.  Such operators are said to be
\textbf{normal}.  The eigenvalues of a normal operator need not be
real, but nevertheless we will discover that $M$ is \textbf{unitarily
  diagonalizable}.

Unwrapping this will lead us to consider again the \textbf{Schur
  decomposition}, which we saw briefly on a previous problem set.
Instead of thinking about orthogonal complements (which is how I
expected you would previously think about the Schur decomposition),
this week we will reprise the Schur decomposition but via the
\textbf{quotient spaces} we discussed last week.

There is also a connection to the \textbf{Fundamental Theorem of
  Algebra}.  To connect that theorem to linear algebra, I encourage
you to read Harm Derksen's article \textit{The Fundamental Theorem of
  Algebra and Linear Algebra} in \textit{The American Mathematical
  Monthly}, Vol.~110, No.~7 (Aug.--Sep., 2003), pp. 620--623.
As the course wraps up, we see many threads coming together, and we
are increasingly making connections between previous topics.  Both
linear algebra and differential equations are broad topics, and it is
remarkable that a topic can be at once big but also tightly connected!

\end{document}

